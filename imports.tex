
%%%%%%%%% Packages used in sample chapters (not otherwise required) %%%%%%%

%% Package for code listing in Appendix A.
\usepackage{listings}%   documentation is here https://ctan.org/pkg/listings

%% Latin filler used in Chapter 1, with a test for package version date. https://ctan.org/pkg/lipsum
\usepackage{lipsum}
\IfPackageAtLeastTF{lipsum}{2021/09/20}{\setlipsum{auto-lang=false}}{}


%%%%%%%%%  Graphics path (to figure files)  %%%%%%%%%%%%%%%%%%%%%%%%%%%%%%%%

%% Can set graphicspath to point to specific directories containing figures (the current directory is searched automatically)
%% For instance, to search a subdirectory of the current directory called "figures" and a parallel directory called "art", set:

% \graphicspath{ {figures/} {../art/} }% For details see: https://latexref.xyz/dev/latex2e.html#g_t_005cgraphicspath


%%%%%%%%%  Representative set-up for biblatex  %%%%%%%%%%%%%%%%%%%%%%%%%%%%%

%\usepackage[style=ieee,maxbibnames=10,sorting=none]{biblatex}% style=ext-numeric-comp,articlein=false,giveninits=true
	%\DefineBibliographyStrings{english}{url= \textsc{url} ,  }% replaces default "[Online]. Available" by "URL"


%\addbibresource{bibliography.bib}

%% to avoid split urls and stretched white space, you can set the bibliography ragged-right:
%\appto{\bibsetup}{\raggedright}

% biblatex is very powerful, and you can customize most aspects the reference list and citations to suit your needs.
% documentation is here: https://ctan.org/pkg/biblatex


%%%%%%%%%%  Option to use natbib   %%%%%%%%%%%%%%%%%%%%%%%%%%%%%%%%%%%%%%%%%

%\RequirePackage[numbers,sort&compress]{natbib}
 
%%% add bibliography to table of contents
%\apptocmd{\bibliography}{\addcontentsline{toc}{chapter}{\protect\textbf{\bibname}}}{}{}

%%% You can use this to rename the bibliography section
%\renewcommand{\bibname}{References}

%%% Can adjust space between bibliography items (change 4pt to something else; don't drop last two lengths, they are stretchable "glue")
%\setlength\bibsep{4pt plus 1pt minus 1pt}


%%%%%%%%%%  Table related packages  %%%%%%%%%%%%%%%%%%%%%%%%%%%%%%%%%%%%%%%%

\usepackage{booktabs}% better quality tables, https://ctan.org/pkg/booktabs
\usepackage{array}%    additional options for table columns, https://ctan.org/pkg/array

%\usepackage{tabularx}%   https://ctan.org/pkg/tabularx

%\usepackage{dcolumn}%    alignment on decimal place, https://ctan.org/pkg/dcolumn
%\newcolumntype{d}[1]{D{.}{.}{#1}}


%%%%%%%%%%  Option for "double spacing" %%%%%%%%%%%%%%%%%%%%%%%%%%%%%%%%%%%%

%% Back in the typewriter era, double spaced lines were convenient for editing with a pencil. 
%% In typography, the separation between lines is called "leading", and it is usually set in 
%% proportion to the font size (i.e., when the font is loaded).  If you really feel the need 
%% to change the line separation, the most attractive results will be obtained by changing the
%% leading in proportion to the the current font size, rather than just doubling the space.

%% The setspace package provides a tool for changing line separation. Use these two commands here:
%
% \usepackage{setspace}%  documentation at https://ctan.org/pkg/setspace
% \setstretch{1.1}% you can choose some other value for the stretch of space between lines
%
%% Use one or more of the these commands AFTER the frontmatter
%
% \onehalfspacing
% \doublespacing
% \singlespacing  % will turn these effects off (you can use these anywhere in the document)

%% The best result may be to stay with leading selected by the typographer who set up the font.

\usepackage{import}
\usepackage{tabularx}
\usepackage{mathtools}
\usepackage{booktabs}
\usepackage{multirow}
\usepackage{siunitx}
\usepackage{tikz}
% Include other packages here, before hyperref.
\usepackage{graphicx}
\usepackage{amsmath}
\usepackage{amssymb}

\usepackage{comment}
\usepackage{color}
\usepackage{svg}
\usepackage{subcaption}
\usepackage{algorithm}
\usepackage{algpseudocode}
\usepackage{upgreek}
\usepackage{stackengine}
\usepackage{imakeidx}	% for index
\makeindex[columns=3]
\usepackage{amsthm}
\usepackage{authoraftertitle}

\usepackage[noabbrev,capitalize]{cleveref}
%\crefname{section}{Sec.}{Secs.}
%\crefname{section}{\textsection}{Secs.}
\crefformat{section}{\textsection#2#1#3}
\Crefname{section}{\S}{\S}
\crefformat{subsection}{(\textsubsection#2#1#3)}
\Crefname{subsection}{\S}{\S}
\Crefname{table}{Table}{Tables}
\crefname{table}{Tab.}{Tabs.}
\crefname{sidenote}{SN}{sn}
\crefname{equation}{Eq.}{Eqs.}

\usepackage{titlecaps}
\Addlcwords{are or for of as into a on}

\makeatletter
\def\cref@getref#1#2{%
  \xdef\@lastusedlabel{#1}%
  \expandafter\let\expandafter#2\csname r@#1@cref\endcsname%
  \expandafter\expandafter\expandafter\def%
    \expandafter\expandafter\expandafter#2%
    \expandafter\expandafter\expandafter{%
      \expandafter\@firstoftwo#2}}%
\creflabelformat{section}{#2\S\nameref*{\@lastusedlabel}#3}
\makeatother
