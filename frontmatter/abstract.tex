\chapter*{Abstract}
\addcontentsline{toc}{chapter}{Abstract}
The rapid advancements in Artificial Intelligence (AI) and computer vision have revolutionized numerous fields while contemporaneously posing significant challenges, particularly in data acquisition and energy consumption. As AI models grow in complexity and size, the environmental impact and monetary costs of their training processes and the demand for extensive, high-quality datasets become critical concerns. At the same time, data scarcity in some fields can impede model generalization and reduce the effectiveness of AI models in real-world applications. 

This dissertation explores these challenges through the lens of medical AI, particularly in ophthalmology. It proposes four innovative solutions spanning the entire machine learning pipeline, from data acquisition and labeling to applying a model in a practical setting. These contributions aim to alleviate the pressures associated with training data-intensive models and to provide a framework for harnessing AI's potential while minimizing its drawbacks.

The first part focuses on budget allocation for image labeling. It introduces an approach to determine annotation strategies for segmentation datasets, estimating the optimal distribution of segmentation and classification labels within a fixed budget. Then, in the second part, the focus is shifted to the issue of domain adaptation for semantic segmentation. Initially, through the introduction of an adapter for the Segment Anything Model (SAM) that facilitates broad generalization and effective test-time domain adaptation. Additionally, a method is presented for the unsupervised setting of semantic segmentation. Lastly, the problem is approached from a clinical standpoint, proposing a method for biomarker localization in Optical Coherence Tomography (OCT) with minimal annotations, using a loss function that ensures biologically plausible outcomes.

Beyond the specific field of ophthalmology, the insights and solutions presented here have broader implications for medicine and other disciplines, such as planetary sciences. Contrary to the notion that AI will replace radiologists, this work envisions a future where AI acts as a complementary tool, enhancing human expertise rather than supplanting it. For such a future to materialize, advancements in AI performance---like the ones presented here---and explainability are essential, alongside the development of robust regulatory frameworks that safeguard both clinicians and patients.

In conclusion, while the challenges posed by AI are substantial, this work offers a hopeful perspective, emphasizing that with thoughtful solutions, the benefits of AI can be maximized to drive societal progress and improve healthcare outcomes.

