\graphicspath{{ch9_futurework/}{Figures/}}

\chapter{Looking Ahead}
\label{chapter:future_work}

% Contents: 
% Labeling: 
%   Adaptive strategy, with a mean dependent on the surface. 
%   Application to more types of labels.
%   Application to the same type of annotations but with different levels of expertise
%   Automated labeling 
% Continual generalization?
%   Preventing forgetting
%   Test-time
% RL applied to computer vision
% Synthetic data


\sidechaptersummary{How Limitations take to New Ideas, Test-time Prediction and its Connection to Human Thinking, Reinforcement Learning in Computer Vision}

\subsubsection{Synopsis} It is difficult to predict the future of a scientific discipline. One can foresee future lines of work from a right assessment of the current methods and theories' limitations or from bridging two fields. By critically evaluating the existing methodologies, we can identify the gaps and challenges that must be addressed to advance the field. This assessment provides a roadmap for future research, highlighting areas where incremental improvements can lead to significant breakthroughs. In this section, I will dive into the limitations of the methods outlined in the dissertation. Then, I will allow for some mind-wandering that will take me to venture into potential bridges that might be interesting to pursue. 


% Labeling: 
%   Adaptive strategy, with a mean dependent on the surface. 
%   Application to more types of labels.
%   Application to the same type of annotations but with different levels of expertise
%   Automated labeling 

\section{Improve Labeling to Reduce Data Dependency}
An objective assessment of a current technique's limitations may sprout new avenues for research. \Cref{chapter:fullweak} proposed a method for online budget-aware data labeling, and the results already pointed in a direction for future work. In the first place, we identified that our method was greedy. Secondly, we assumed that segmentation performance increases logarithmically with the volume of data. This assumption, backed by previous research\sideauthorcite{sun2017}, was utilized to set the mean prior for the GP, a fundamental piece of the algorithm. Even if it is unclear whether this asymptotic logarithmic trend assumption is violated for one of the evaluation datasets, we believe that some additional learnable parameters should be added to the mean prior to reflect the bizarre behavior of this dataset. Out of the four, this is the only dataset in which the rates of performance improvement\sidenote{For segmentation and classification.} increase with the number of samples. Our method is not equipped with anything to face this situation, hence its failure.

Furthermore, there is an immediate following for this work that might prove interesting. Even if the method should be agnostic to the annotation type, both for weak and full, studying its behavior with other combinations could be fruitful in understanding which annotation types are more cost-efficient. Aside from this, an expansion to more dimensions, say image-wise classification labels, bounding boxes, and segmentation labels altogether, would only be a challenge in computing power. The combinations in this scenario would grow exponentially with respect to the 2D case that we proposed.

Another interesting line of work related to annotation strategies involves the integration of synthetic labels into the game. The recent advancements in computer graphics are permeating the computer vision field by introducing increasingly realistic synthetic datasets. Synthetic data allows for a more controlled and predictable environment, which greatly simplifies the labeling process\sidenote{For instance, acquiring normal maps in natural data requires depth estimation. In synthetic data, they come naturally from the scene.}. Furthermore, generated images can provide the model with scenarios that are difficult to retrieve, such as dangerous situations for self-driving or highly rare medical conditions. However, whether a model can be fully trained from synthetic data remains unclear. Therefore, the combination of natural and synthetic data will likely yield the best results. Determining the optimal ratio will be challenging, but I anticipate that methods similar to those presented in this work will play a key role in this field. 

I have reiterated in this thesis the significant investment required to acquire expert annotations. Nevertheless, it can be argued that no two experts are identical. Individuals who have undergone professional training and possess considerable experience are typically compensated more than those who have recently graduated. Conversely, it is assumed that the former group possesses greater knowledge than the latter due to their experience. This could be modeled in a similar framework to that described in \Cref{chapter:fullweak}, where the labeling modality combination would be between senior and junior annotators, with a non-unitary cost ratio. Domain knowledge and experience could be incorporated by introducing noise to one of the axes.  

% Continual generalization?
%   Preventing forgetting
%   Test-time

\section{Generalizing Models}\index{catastrophic forgetting}
Current domain adaptation networks often face the significant challenge of not retaining information about the source domain, leading to a phenomenon known as catastrophic forgetting. If we strive to have AI assistants more present in routine tasks, it is imperative to develop advanced continual learning techniques that enable models to maintain and integrate knowledge from all domains seamlessly. Image acquisition devices will improve, leading to higher-resolution images with better contrast. However, the intrinsic knowledge of deep learning models should be maintained. One-year-old kids can discern a lion from a tiger on TV even if they have seen only drawings of them, and this is because they know what a lion \textit{is} and not only how it \textit{looks}. Artificial intelligence models should be capable of generalizing and adding knowledge to what they already \textit{know}. The traditional trend of training larger models from scratch every few months is not sustainable and will become obsolete in the short to medium term. Resources should be invested in researching alternatives to this paradigm.

In this regard, test-time domain adaptation methods offer an intermediate solution because they let the model work on a prediction and determine if that is the best prediction it can produce. As contrary as I am to the anthropomorphization of deep learning models and their identification with the human brain, I think it is reasonable to think of test-time domain adaptation algorithms in terms of the thought process we humans do when we are exposed to novel information. Without realizing it, our brain processes and works on the information, not letting the first thing that comes to be deployed through our mouth. Test-time algorithms work in this direction. Therefore, I believe they will gain popularity in the future.

In this line, I believe applying reinforcement learning-like techniques to computer vision, as it is done nowadays to natural language processing, will become beneficial. Segmentation models can already estimate how accurate their predictions are with high precision. Hence, this information could be used as self-supervision for the next iteration. Of course, the space of semantic segmentations is much more varied than a vocabulary, given the amount of pixels in an image. At the same time, the possible valid segmentations are less varied than valid sentences one can generate from a prompt. Generating varied predictions such that exploration is done in a meaningful way is, therefore, a complicated task. 