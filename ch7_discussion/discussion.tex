\graphicspath{{ch7_discussion/}{Figures/}}

\chapter{Looking back}
\label{chapter:discussion}

% Contents: 
% (Talk about the thesis, how it follows the machine learning pipeline)
% Rethinking Data Labeling:
%   Standard data labeling strategies and fine-tuning
%   Our approach for data labeling
% Adapt to unseen data:
%   Large models need less adaptation
%   Into the unsupervised
% Weak labels for a real-world problem
% Limitations (for each section?)


\sidechaptersummary{Summary of Contributions, Link to the Thesis Statement}
\desctotoc{Summary of Contributions --- Link to the Thesis Statement}

\subsubsection{Synopsis} Technology advancements are transforming society, yet they also bring novel challenges that must be addressed for the transition to be seamless for all society. One of the most urgent challenges lies in the unavailability of large datasets to feed the ever-growing models in fields where data access is limited. The previous four chapters have outlined four independent solutions to alleviate this pressure, each addressing a unique facet of the problem.

This chapter will be the final integration into a cohesive dissertation, linking each contribution to the central thesis statement. 

\sectionlinenew

We began this dissertation with a rather pessimistic perspective on the impact of AI on the contemporary world. Technology is advancing at an unstoppable pace, faster than ever before, and this brings with it new problems that must be addressed. We are facing a transformation akin to the Industrial Revolution back in the 18th and 19th centuries, with similarly transformative potential consequences for society and the economy\sideauthorcite{abis2024changing}. The question is whether we will be able to replicate the growth in the standard of living that the late modern era witnessed upon the development of machines\sideauthorcite{nardinelli2008industrial}.

Just as society has evolved, so too have the obstacles that progress carries. The current challenges include the environmental impact of training large models, the ethical biases inherent in the data, and the necessity for large, high-quality datasets that can feed the data-hungry models. This thesis has presented an optimistic view on the latter of these aspects, offering four solutions that span the entire machine learning pipeline. These solutions have been designed to verify the thesis statement introduced in \Cref{chapter:introduction}. As this statement serves as the foundation for the present work, I will repeat it here for the reader's convenience:
\\

\textit{The surging demand for larger datasets, promoted by ever-larger models and the advancements in computing, can be counteracted by carefully allocating the budget, making informed labeling choices, and leveraging field-specific constraints to the model's advantage.}
\\

The following sections will present a detailed examination of how each chapter has contributed to verifying this statement. In \Cref{sec:disc_fullweak}, we will see that a strategic budget allocation can contribute to reducing data labeling costs. In \Cref{sec:disc_adapt}, we will focus on model training and see that large models are proficient at generalization. Finally, \Cref{sec:disc_oct} will highlight the importance of field-specific constraints to reduce the necessity of strongly annotated data. %We will then conclude with some final remarks and the limitations of the present work. 

% Rethinking Data Labeling:
%   Standard data labeling strategies and fine-tuning
%   Our approach to data labeling

\section{Rethinking Data Labeling}
\label{sec:disc_fullweak}

The preparation of data represents a fundamental stage within the machine learning pipeline. It is frequently assumed that the availability of a vast quantity of data is indispensable for the successful implementation of any machine learning application\sidenote{In the end, there is this field called \textit{Big Data}, right?}. However, data quality is of even greater importance, as discussed in \yeartextcite{widner2023lessons}. On the same line, the statistician Xiao-Li Meng demonstrated in a 2016 presentation that without data quality, very large quantities of data are required to reduce bias\sideauthorcite{meng2016statistical}. There is, however, one variable that research studies often disregard: monetary budget. As we saw in \Cref{fig:supervision_prices}, high-quality labeling raises the cost of a project. The question then turns into whether we can increase the performance of a model by using lower-quality data\sidenote{In this context, lower-quality data means \textit{more weakly annotated}.} while keeping the budget constant.

We address this topic in \Cref{chapter:fullweak}, where we focus on the task of image semantic segmentation. Building a segmentation dataset is known to be time-consuming, and the procedure of producing ground-truth segmentation annotations is still a tedious, manual task. When budget enters the game, traditional machine learning practitioners rely on fully supervised fine-tuning to achieve the best possible results, often from a pretrained model in a different domain. Given the expensive annotation costs, this approach incurs a reduced amount of different information fed to the model, which can promote overfitting. This problem is accentuated in fields that require experts to annotate the data, such as medicine. High-quality, annotated data in this field is scarcer.

\subsection{Our Approach to Data Labeling}
Being aware of the previously mentioned constraints in the medical field, we hypothesize in \Cref{chapter:fullweak} whether it is possible to combine weak and full annotations and, if so, how to do it. By definition, weak annotations are cheaper than full annotations. Consequently, with a fixed budget, more images can be labeled. This seemingly simple leap provides the model with more data, which serves to reduce overfitting and enhance performance on previously unseen samples. However, determining an appropriate allocation strategy for the annotation budget is a challenging task, as there is no guarantee that the same strategy will be shared in different datasets or even models. At the same time, it could be argued that the need for weakly annotated samples decreases as the budget increases because the model can retrieve all the necessary information from fully annotated samples. This implies that the allocation strategy is not constant but should depend on the budget.

We propose a method in \Cref{chapter:fullweak} for determining the optimal budget allocation strategy. Taking the previous ideas into account, it is evident that the method must work iteratively, alternating between partial budget allocation, label acquisition from an oracle, and model training. In essence, we estimate at each iteration how different annotation choices will affect model performance and move in the direction of the Pareto optima between expected improvement and cost. Without loss of generality, we identify weak labels with image-level labels and full labels with semantic segmentation maps. With this, we implement a straightforward training schedule consisting of training first for classification and then for segmentation, in both cases with fully-supervised loss functions. 

Our method requires an initial strategy to initiate the iterative process. This involves defining a number of classification and segmentation labels according to an initial associated cost. New data is annotated and added to the available pool at each iteration. It is then sampled to fit a Gaussian Process (GP), which is used to define a surrogate function for the utility of an allocation strategy. The surrogate utility function is then employed to estimate the next best strategy for the next iteration.

We evaluate our method against fixed policies representing the budget ratio allocated to annotating for segmentation or classification. Aside from our method, the experiments confirm the previous claims: best-performing strategies differ per dataset and budget. Therefore, choosing a fixed strategy blindly is, on average, a suboptimal choice. Regarding our method, the results show that it can consistently produce good performances across different budget ratios and datasets. We attribute the positive result to the iterative nature of the method, which allows it to adapt to the budget scenario. However, we also recognize that our method has limitations. First, the GP is heavily influenced by the choice of the mean prior. Not all datasets are guaranteed to behave similarly\sidenote{\textit{I. e.}~logarithmically, which we enforced in the mean prior.} with new annotations, and this behavior is only appreciated at the end of the data labeling process. Therefore, an adaptive mean prior might be needed if this method is taken to a real scenario. Second, because of the definition of the utility function, the method is greedy in nature, meaning that its adaptation capacity is limited. It will show a tendency to follow the current best-performing strategy.

With this work, we have seen that the demand for larger annotated datasets can be counteracted by strategically allocating the available budget. For low-budget regimes, carefully allocating this budget is more profitable than blindly choosing a fixed policy. These findings are relevant to our main field of study --- medical imaging ---, where annotated samples are scarce.


% Adapt to unseen data:
%   Large models need less adaptation
%   Into the unsupervised

\section{Adapt to Unseen Data}\label{sec:disc_adapt}
\Cref{chapter:samda,chapter:tist} dealt with the domain adaptation problem in two different ways. First, \Cref{chapter:samda} proposed a training method for large models that achieved better generalization by training only a small subset of the parameters. Then, \Cref{chapter:tist} addressed unsupervised domain adaptation by leveraging invariance to image transformations.

\subsection{Large Models Need Less Adaptation}
Training large models is a challenging engineering task. Not only do they need a lot of data, but one also has to consider the limited memory of the GPUs where they are stored. Nowadays, many large models have grown beyond a single GPU. This has sparked new research on model and data parallelism to increase the number of GPUs during training. At the same time, some efforts have been directed to mixed precision training\sideauthorcite{micikevicius2017mixed} to nearly halve the memory consumption of deep learning models during training.

In parallel, the method we proposed in \Cref{chapter:samda}, along with other PEFT techniques, tries to reduce the models' trainable parameters, assuming that large models already have a sufficient representation of the world that can be adapted to niche areas that need further fine-tuning. In that chapter, we developed an adapter for the Segment Anything Model (SAM)\sideauthorcite{sam}. This adapter leveraged a pretrained model that inherently contained domain knowledge. As a result, neither the image encoder nor the mask decoder required substantial parameter updates during the adaptation phase. This design choice significantly reduced the number of trainable parameters to only an overhead of 1\% of the parameters of the whole network.

In order to evaluate the efficacy of our method, we compared it against other PEFT baselines on three tasks: in-domain, fully supervised semantic segmentation, out-of-domain generalization, and test-time domain adaptation. For the out-of-domain tasks, we utilized three medical datasets comprising images captured from two distinct devices, creating two domains, thereby introducing a domain shift. In the case of the in-domain task, we also added a natural image dataset. The base vision transformer in SAM was always initialized with pretrained weights, whether from larger medical datasets or the official pretraining weights. The results demonstrated that training with the adapter was superior at generalizing precisely because it leveraged the domain knowledge rooted in the neural network. It also exhibited superior performance in test-time domain adaptation due to the small number of trainable parameters it encompassed. Finally, the performance gain against full fine-tuning decreased as the number of training images increased. This was an expected outcome: with more data samples, the training distribution widened. Because the adapter's capacity was much smaller than that of the whole model, it could not adapt to wide distributions.

These results suggest that adapting a large model to one's needs is possible. The vision transformer we used in \Cref{chapter:samda} is only the base one, with 86M parameters. We believe comparable results could be obtained with larger models using a similar adaptation approach, assuming that the quality of the pretraining weights also increases. 

\subsection{Into the Unsupervised}
\Cref{chapter:tist} addressed the adaptation problem differently. Instead of relying on a large model to extract knowledge, this chapter focused on unsupervised domain adaptation for semantic segmentation, a problem in which no ground-truth labels are available for the target domain. As previously discussed in \Cref{sec:domain_adaptation}, this problem resembles a real-world application. In a real scenario, having semantic segmentation labels for each device and configuration would be implausible, especially in highly specific fields like medicine, where expertise is essential. Semi-supervised learning has sought to reduce annotation requirements in the target domain, involving methods that promote learning abstract representations from an unlabeled target set and extending decision boundaries to align with the target dataset distribution.

The proposed method employed transformation-invariant, highly-confident pixel predictions in the target dataset for self-training. This was accomplished using an ensemble of high-confidence predictions from various non-spatially transformed versions of the same input. The loss function comprised two terms: in the first place, a regularization cross-entropy term between the source domain labels and their corresponding predictions. In the second place, it included a pseudo-supervised branch acting on the most confident pixels between the target predictions before and after transformations. The contribution of the second term was increased progressively as training progressed. The evaluation against six alternative methods for unsupervised domain adaptation across three datasets and two neural network architectures demonstrated that pseudo-label training with high-confidence pixels was superior on average to other alternatives.  

In the context of the whole thesis, \Cref{chapter:tist} demonstrated that semi-supervised learning is a powerful strategy for those scenarios where labels are not provided. In my opinion, domain adaptation problems in which target domains are unavailable\sidenote{Unsupervised DA is one of them, but also test-time and source-free.} lies at the top of the difficulty. In these problems, the model has to learn meaningful representations without supervision labels and extract the intrinsic knowledge inside the data, discarding unimportant and distracting features.

The results in the mentioned chapter, as well as in several published works cited throughout this thesis, prove that domain adaptation without target labels can be achieved. This realization alleviates the pessimistic perspective that started the present chapter: the need for labeled data might be less pressing than was delineated at the beginning of \Cref{chapter:introduction}. Instead, complementary efforts should be (and are) taken toward a better comprehension of the inner workings of deep learning models. This would help us develop more capable adaptation methods.
% Weak labels for a real-world problem
\section{Weak Labels and Domain Knowledge for a Real-World Problem}\label{sec:disc_oct}

The last sections have covered data acquisition and model adaptation, but \Cref{chapter:oct} focuses on a problem closer to a real-world application. Reading Centers\sidedef{Reading Center}{\jgt{TODO}} across the world work tirelessly to classify thousands of ophthalmic images. They screen one image after another to find biological markers for potential malignant conditions in patients following clinical trials. Their employees are specifically trained for this purpose, and most of their work is manual. AI could alleviate the laborious screening task, and serve as a triage for those images that do not require a human expert to review them. With a growing and ever-older population, the number of ophthalmic images is only going to increase in the foreseeable future. Automatic detection methods will, at that point, be of invaluable help.

\Cref{chapter:oct} is our effort to ease this problem in OCTs. We propose a method that locates biological markers in ETDRS rings, which only requires weak labels in the form of slice-level binary annotations of marker presence during training. This way, we avoid the need for expensive segmentation annotations. During testing, our method infers marker presence and location in ETDRS rings. We achieve this via two combined mechanisms: first, we realize that the conversion from localization within an image to an ETDRS ring is a mere post-processing step that depends solely on the slice position within the C-scan. Accordingly, we pool each image's feature map in columns, preserving spatial relations. Each one of these columns is classified independently with the same MLP. At the end, we perform the post-processing step, converting the column predictions into ETRDS rings.

On the other hand, we enforce geometrically and biologically plausible solutions with a tailored loss function. The previous column classification, combined with slice-level annotations, gives enough information about the location of the markers. When a marker is not present --- \ie~its label is negative --- it cannot be present in any of the columns. At the same time, if a marker is present --- \ie~its label is positive --- it must be present in at least one of the columns. 

We test our method and compare it against a range of alternative approaches. In general, our method consistently outperforms the benchmark alternatives across all marker and ETDRS ring combinations. This supports our hypothesis that feature maps can be effectively utilized to identify marker locations at a coarse level. Moreover, we assess the efficacy of our method prior to the post-processing step with the \autoindex{en-face projection}, showing positive results.

This chapter builds on the conclusions drawn in the previous section. We see that weak annotations are sometimes sufficient to achieve relevant results, provided that other characteristics of the data or the domain are taken into account. This reiterates a previous idea: weak annotations are more cost-effective to retrieve, allowing an equivalent budget to access a greater quantity. Their significance should not be disdained, as deep learning models benefit from the exposure to a diverse range of data samples that cover a broad distribution.