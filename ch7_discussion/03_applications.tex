% Weak labels for a real-world problem
\section{Weak Labels and Domain Knowledge for a Real-World Problem}\label{sec:disc_oct}

The last sections have covered data acquisition and model adaptation, but \Cref{chapter:oct} focuses on a problem closer to a real-world application. Reading Centers\sidedef{Reading Center}{Facility that evaluates and interprets retinal images for clinical trials and studies.} across the world work tirelessly to classify thousands of ophthalmic images. They screen one image after another to find biological markers for potential malignant conditions in patients following clinical trials. Their employees are specifically trained for this purpose, and most of their work is manual. AI could alleviate the laborious screening task, and serve as a triage for those images that do not require a human expert to review them. With a growing and ever-older population, the number of ophthalmic images is only going to increase in the foreseeable future. Automatic detection methods will, at that point, be of invaluable help.

\Cref{chapter:oct} is our effort to ease this problem in OCTs. We propose a method that locates biological markers in ETDRS rings, which only requires weak labels in the form of slice-level binary annotations of marker presence during training. This way, we avoid the need for expensive segmentation annotations. During testing, our method infers marker presence and location in ETDRS rings. We achieve this via two combined mechanisms: first, we realize that the conversion from localization within an image to an ETDRS ring is a mere post-processing step that depends solely on the slice position within the C-scan. Accordingly, we pool each image's feature map in columns, preserving spatial relations. Each one of these columns is classified independently with the same MLP. At the end, we perform the post-processing step, converting the column predictions into ETRDS rings.

On the other hand, we enforce geometrically and biologically plausible solutions with a tailored loss function. The previous column classification, combined with slice-level annotations, gives enough information about the location of the markers. When a marker is not present --- \ie~its label is negative --- it cannot be present in any of the columns. At the same time, if a marker is present --- \ie~its label is positive --- it must be present in at least one of the columns. 

We test our method and compare it against a range of alternative approaches. In general, our method consistently outperforms the benchmark alternatives across all marker and ETDRS ring combinations. This supports our hypothesis that feature maps can be effectively utilized to identify marker locations at a coarse level. Moreover, we assess the efficacy of our method prior to the post-processing step with the \textit{en face} projection\index{en face projection}, showing positive results.

This chapter builds on the conclusions drawn in the previous section. We see that weak annotations are sometimes sufficient to achieve relevant results, provided that other characteristics of the data or the domain are taken into account. This reiterates a previous idea: weak annotations are more cost-effective to retrieve, allowing an equivalent budget to access a greater quantity. Their significance should not be disdained, as deep learning models benefit from the exposure to a diverse range of data samples that cover a broad distribution.