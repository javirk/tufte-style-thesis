\graphicspath{{ch5_tist/}{figures/}}
% \graphicspath{{Figures/}}

\chapter{Domain Adaptation for Medical Image Segmentation using Transformation-Invariant Self-Training}
\chaptermark{Domain Adaptation with Transformation-Invariant Self-Training}
\label{chapter:tist}

\sidechaptersummary{Unsupervised Domain Adaptation Method, Transformation Invariant Predictions}
\desctotoc{Unsupervised Domain Adaptation Method --- Transformation Invariant Predictions}

\subsubsection{Synopsis}Models capable of leveraging unlabelled data are crucial in overcoming large distribution gaps between the acquired datasets across different imaging devices and configurations. In this regard, self-training techniques based on pseudo-labeling have been shown to be highly effective for unsupervised domain adaptation. However, the unreliability of pseudo labels can hinder the capability of self-training techniques to induce abstract representation from the unlabeled target dataset, especially in the case of large distribution gaps. 
Since the neural network performance should be invariant to image transformations, we look to this fact to identify uncertain pseudo labels. Indeed, we argue that transformation invariant detections can provide more reasonable approximations of ground truth. Accordingly, we propose an unsupervised domain adaptation strategy termed transformation-invariant self-training (TI-ST) to assess pixel-wise pseudo-labels' reliability and filter out unreliable detections during self-training. We perform comprehensive evaluations for domain adaptation using three different modalities of medical images, two different network architectures, and several alternative state-of-the-art domain adaptation methods. Experimental results confirm the superiority of our proposed method in mitigating the lack of target domain annotation and boosting segmentation performance in the target domain.

\subsubsection{Publication}This chapter is based on a publication\sideauthorcite{ghamsarian2023domain}, and its contents have been modified slightly to be more consistent with the rest of the thesis. In particular, the notation in \Cref{sec:tist_methodology} has been adapted to agree with the rest of the thesis, \Cref{fig:tist_method} has been modified, and \Cref{fig:tist_ablation,fig:ablation_stability} have different colors to those in the published version.

\subsubsection{Author contributions}The work in this chapter was done in collaboration with the University Hospital Bern. The contributing authors were Negin Ghamsarian, Pablo Márquez Neila, Sebastian Wolf, Martin Zinkernagel, Klaus Schoeffmann, and Raphael Sznitman. My contribution consisted of helping with running the experiments and building the figures for the final version of the paper.

% \graphicspath{{thesis-source/}{ch3_oct_localization/}{Figures/}}
\section{Introduction}
\label{sec:oct_introduction}
\autoindex{Age-Related Macular Degeneration} (AMD) and \autoindex{Diabetic Retinopathy} (DR) are two of the most common eye diseases, with over 300 million patients at risk of losing sight worldwide\sidecite{Bourne2017}. To diagnose and manage these chronic retinal conditions, 30 million \autoindex{Optical Coherence Tomography} (OCT) are taken each year, yielding micron-resolution 3D volumes of the retina in a routine, fast, and noninvasive way. OCT has become a crucial instrument for establishing patient treatments and a dependable tool to validate the efficacy of novel therapeutic approaches to treat eye diseases.

In this context, \autoindex{intraretinal fluid} (IRF) and \autoindex{subretinal fluid} (SRF) are well-established markers that are directly linked to both AMD and DR\sidecite{Phadikar2017,zur2017}. Their identification and localization within a set of concentric rings, known as the \autoindex{Early Treatment Diabetic Retinopathy Study} (ETDRS) rings\sidecite{Domalpally2018}, is critical for disease assessments\sidenote{See~\cref{fig:etdrs_rings}}, as the different ETDRS ring regions are linked to different visual function levels (\ie,~higher risk of vision loss when markers are in the central 1mm ring and lower risk when in the 6mm ring). Driven by this clinical need, numerous methods have been proposed to automate the process of identifying markers such as IRF and SRF\sidecite{Trucco2020}, and the work here follows this research direction too.
%\begin{figure*}[t]
%%\centering
%\includegraphics[width=\textwidth]{Figures/fig1_SLO.pdf}
%\sidecaption{Left: View of the retina, the OCT volume (green square) and the ETDRS rings (white) which are virtually placed on the surface of the retina. Right: Three 2D OCT slices at different positions of the OCT volume.
%Red circles indicate IRF biological marker and the yellow rectangle indicates SRF (Figure best seen in color).
%}
%\label{fig:etdrs_rings}
%\end{figure*}

%\plainwidefig{1}{Figures/fig1_SLO.pdf}{Left: View of the retina, the OCT volume (green square), and the ETDRS rings (white) which are virtually placed on the surface of the retina. Right: Three 2D OCT slices at different positions of the OCT volume. Red circles indicate IRF biological marker and the yellow rectangle indicates SRF (Figure best seen in color).}{fig:etdrs_rings}


Previous methods have included IRF and SRF detection and segmentation models\sidecite{DeZanet2020,Lee2018,Liefers2021,Yim2020}. While segmentation models have the advantage of quantifying IRF and SRF regions, they often require a large amount of manually annotated segmentation labels for optimal performance.\index{weakly supervised semantic segmentation} To counteract this issue, some works use weak annotations, such as slice level labels, retinal layer positioning, and foveal distance, to achieve voxel-wise segmentations\sidecite{schlegl2015predicting}. Weak annotations offer a wide range of possibilities, and therefore others have studied the use of bounding boxes to develop positive-aware lesion detection networks\sidecite{Fan2020}. More relevant to our work, some methods only use slice-level annotations\sidecite{ma2020}. Here, Ma et al. presented a weakly-supervised segmentation method for Geographic Atrophy (GA) lesions in Spectral Domain OCT images.  The method first segments the retinal pigment epithelium and then extracts a class activation map from multi-scale features. The final en-face binary segmentation of GA is obtained by refining the map with Conditional Random Fields, utilizing only slice-level labels with binary information about the presence of GA.

\plainwidefig[t]{1}{Figures/fig1_SLO.pdf}{Left: View of the retina, the OCT volume (green square), and the ETDRS rings (white) which are virtually placed on the surface of the retina. Right: Three 2D OCT slices at different positions of the OCT volume. Red circles indicate IRF biological marker and the yellow rectangle indicates SRF (Figure best seen in color).}{fig:etdrs_rings}

Similarly, ensembles of Convolutional Neural Networks (CNNs) have been proposed to detect IRF and SRF in individual slices using only binary annotations on a slice level\sideauthorcite{Kurmann2019,Kurmann2019a}. However, by removing the need for segmentation annotations, these methods cannot provide any location information.  In this work, we propose a novel weakly supervised deep learning framework that overcomes these limitations and enables the detection and localization of 2D OCT markers in ETDRS rings without requiring costly location information during training. Specifically, our method uses binary annotations of marker presence in OCT slices during training and infers marker presence and marker location in ETDRS rings during test time. To do this, we introduce a pooling strategy where we treat our network's convolutional feature maps in such a way as to preserve spatial relations that can be partially pooled for coarse localization. This is combined with a novel loss function that enforces geometrically and biologically plausible solutions. This allows ring assignment to be performed as a post-processing step independent of the training phase. Our experiments demonstrate that our method predicts the location of markers in ETDRS rings with high accuracy, thereby significantly outperforming previous methods that use the same amount of training information. 
\section{Methodology}
\label{sec:tist_methodology}

Consider a labeled source dataset, $\mathcal{S}$, with training images $\mathcal{X_S}$ and corresponding segmentation labels $\mathcal{Y_S}$, while we denote a target dataset $\mathcal{T}$, containing only target images $\mathcal{X_T}$. We aim to train a network using $\mathcal{X_S}$, $\mathcal{Y_S}$, and $\mathcal{X_T}$ for semantic segmentation in the target dataset. 


We propose to train the model using a self-supervised approach on the images $\mathcal{X_T}$ by assigning pseudo labels during training. Typical pseudo labels are computed from independent predictions of unlabeled images. Instead, our proposed framework adopts a self-assessment strategy to determine the reliability of predictions in an unsupervised fashion. Specifically, we propose to target highly-reliable predictions generated by a network aiming for transformation-invariant confidence. Compared to self-ensembling strategies that penalize the distant predictions corresponding to the transformed versions of identical inputs, our goal is to filter out transformation-variant predictions. Indeed, our method reinforces the ensemble of high-confidence predictions from two versions of the same target sample. Our proposed TI-ST framework simultaneously trains on the source and target domains, so as to progressively bridge the intra-domain distribution gap. \cref{fig:BD} depicts our TI-ST framework, which we detail in the following sections. 

%\begin{figure}[t]
%\centering
%\includegraphics[width=\textwidth]{figures/BD8.pdf}
%\caption{Proposed unsupervised domain adaptation framework based on transformation-invariant self-training (TI-ST). Ignored pseudo-labels during unsupervised loss computation are shown in turquoise.
%}
%\label{fig:BD}
%\end{figure}

\plainwidefig{1}{figures/BD8.pdf}{Proposed unsupervised domain adaptation framework based on transformation-invariant self-training (TI-ST). Ignored pseudo-labels during unsupervised loss computation are shown in turquoise.}{fig:BD}

\subsection{Model} At training time, images from the source dataset are augmented using spatial $g(\cdot)$ and non-spatial $f(\cdot)$ transformations and passed through a segmentation network, $N(\cdot)$, by which the network is trained using a standard supervision loss. At the same time, images from the target dataset are also passed to the network. Specifically, we feed two versions of each target image to the network: (1) the original target image $x_\mathcal{T}$, and (2) its non-spatially transformed version, $\Acute{x_\mathcal{T}} = f(x_\mathcal{T})$. 
% \RS{There is no other mention of $f$ throughout the method. We should probably give some information as to what we use, and why.} 
Once fed through the network, the corresponding predictions can be defined as $\tilde{y_{\mathcal{T}}} = \sigma(N(x_{\mathcal{T}}))$ and $\tilde{\acute{y_{\mathcal{T}}}} = \sigma(N(\Acute{x_{\mathcal{T}}}))$, where $\sigma(\cdot)$ is the Softmax operation. We then define a confidence-mask ensemble as
\begin{equation}
\mathcal{M}_{cnf} = 
Cnf(\tilde{{y_{\mathcal{T}}}})
\odot
Cnf(\tilde{\Acute{y_{\mathcal{T}}}}),
\label{eq: ensemble of confidence}
\end{equation}
\noindent
where $\odot$ refers to Hadamard product used for element-wise multiplication, and $Cnf$ is the high confidence masking function,
\iffalse
\begin{equation}
    \mathcal{M}_{cnf}(\tilde{\acute{y_T}}, \Acute{\tilde{y_T}}) = h(\tilde{\acute{y_T}})\odot h(\Acute{\tilde{y_T}})
    \label{eq: confidence mask}
\end{equation}
\fi
\begin{equation}
    Cnf_{\textsub{$\in (W\times H$)}}(y) =
    \begin{cases}
    0, & \text{if    } \maxH_{\textsub{C}}(y) > \uptau\\
    1, & \text{else.  }
    \end{cases}
    \label{eq: filtering}
\end{equation}
\noindent
where $\uptau \in (0.5,1) $ is the confidence threshold, and $H$, $W$, and $C$ are the height, width, and number of classes in the output, respectively. Specifically, $\mathcal{M}_{cnf}$ encodes regions of confident predictions that are invariant to transformations.
We can then compute the pseudo-ground-truth mask for each input from the target dataset as
\begin{equation}
\hat{\Acute{y_{\mathcal{T}}}} = 
\begin{cases}
\argmaxH_{\textsub{C}} (\tilde{\Acute{y_{\mathcal{T}}}}), & \text{if  } \:\mathcal{M}_{cnf} = 1\\
\text{ignore}, & \text{else.  }
\end{cases}
\end{equation}
\noindent

\subsection{Training}
To train our model, we simultaneously consider both the source and target samples by minimizing the following loss,
\begin{equation}
    \mathcal{L}_{overall} = \mathcal{L}_{Sup}( \tilde{y_\mathcal{S}}, y_\mathcal{S}) + \lambda \Big(\mathcal{L}_{Ps}(\mu (\tilde{\Acute{y_{\mathcal{T}}}}), \hat{\Acute{y_{\mathcal{T}}}})\Big) ,
    \label{eq: loss}
\end{equation}
\noindent
where $\mathcal{L}_{Sup}$ and $\mathcal{L}_{Ps}$ indicate the supervised and pseudo-supervised loss functions used, respectively. We set $\lambda$ as a time-dependent weighing function that gradually increases the share of pseudo-supervised loss. Intuitively, our pseudo-supervised loss enforces predictions on transformation-invariant highly-confident regions for unlabeled images. 

\subsubsection{Discussion} 
The quantity and distribution of supervised data are determining factors in neural networks' performance. With highly distributed large-scale supervisory data, neural networks converge to an optimal state efficiently. However, when only limited supervisory data, with heterogeneous distribution from the inference dataset, using more sophisticated methods to leverage a priori knowledge is essential. Our proposed use of invariance of network predictions with respect to data augmentation is a strong form of knowledge that can be learned through dataset-dependent augmentations. The trained network is then expected to provide consistent predictions under diverse transformations. Hence, the transformation variance of the network predictions can indicate the network's prediction doubt and low confidence correspondingly. We take advantage of this characteristic to assess the reliability of predictions and filter out unreliable pseudo-labels.\jgt{Check all this paragraph}

\section{Experimental setup}
\label{sec:tist_experimental_settings}

\subsection{Datasets}
We validate our approach on three cross-device/site datasets for three different modalities: 

\begin{itemize}
    \item \textbf{Cataract:} instrument segmentation in cataract surgery videos. We set the ``Cat101''~\sidecite{cat101} as the source dataset and the ``CaDIS'' as the target domain dataset~\sidecite{CaDIS}. 
    \item \textbf{OCT:} IRF Fluid segmentation in retinal OCTs~\sidecite{retouch}. We use the high-quality ``Spectralis'' dataset as the source and the lower-quality ``Topcon'' dataset as the target domain.
    \item \textbf{MRI:} multi-site prostate segmentation~\sidecite{liu2020ms}. We sample volumes from ``BMC'' and ``BIDMC'' as the source and target domain, respectively\sidenote{``BIDMC'' imposes a more challenging scenario than ``BMC'' due to the low contrast of the images.}.
\end{itemize} 

We follow a four-fold validation strategy for all three cases and report the average results over all folds. The average number of labeled training images (from the source domain), unlabeled training images (from the target domain), and test images per fold are equal to ($207, 3'189,58$) for Cataract, ($391,569,115$) for OCT, and ($273,195,65$) for MRI dataset.

\subsection{Baseline methods}
We compare the performance of our proposed transformation-invariant self-training (SI-ST) method against seven state-of-the-art alternative methods that represent different domain adaptation methods and paradigms: $\Uppi$ models~\sidecite{TESSL}, temporal ensembling~\sidecite{TESSL}, mean teacher~\sidecite{UDAMIS}, cross pseudo supervision (CSP)~\sidecite{CPS}, reciprocal learning (RL)~\sidecite{Reciprocal} and self-training (ST)~\sidecite{st++}.

\subsection{Networks and training settings}
We evaluate our TI-ST framework using two different architectures:  (1) DeepLabV3+~\sidecite{DeepLabV3} with ResNet50 backbone~\sidecite{ResNet} and (2) scSE~\sidecite{SCSE} with VGG16 backbone. Both backbones are initialized with the ImageNet~\sidecite{deng2009imagenet} pre-trained parameters. We use a batch size of four for the Cataract and MRI datasets and a batch size of two for the OCT dataset. For all training strategies, we set the number of epochs to 100. The initial learning rate is set to 0.001 and decayed by a factor of $0.8$ every two epochs. The input size of the networks is $512\times 512$ for cataract and OCT and $384\times 384$ for the MRI dataset. 
As spatial transformations $g(\cdot)$, we apply cropping and random rotation (up to 30 degrees). 

The non-spatial transformations, $f(\cdot)$, include color jittering (brightness = 0.7, contrast = 0.7, saturation = 0.7), Gaussian blurring, and random sharpening. The confidence threshold $\uptau$ for the self-training framework and the proposed TI-ST framework is set to $0.85$ except in the ablation studies\sidenote{See \Cref{sec:experimental_results_tist}}. In Eq.~\eqref{eq: loss}, the weighting function $\lambda$ ramps up from the first epoch along a Gaussian curve equal to $\exp[-5(1-\text{current-epoch}/{\text{total-epochs}})]$. The unsupervised loss is set to the \autoindex{cross-entropy loss}, and the supervised loss is set to the \textit{cross entropy log dice} loss, which is a weighted sum of cross-entropy and the logarithm of soft dice coefficient. For the TI-ST framework, we only use non-spatial transformations for the self-training branch for simplicity.


\section{Results}
\label{sec:experimental_results_tist}

\begin{table*}[t]
\centering

\caption{Quantitative comparisons in Dice score among the proposed (TI-ST) and alternative methods for DeepLabV3+~\cite{DeepLabV3} (DLV3+) and scSENet~\cite{SCSE} and the three datasets. Relative Dice computed over the Supervised baseline. \label{tab:quantitative}}

% @{}lccccccc@{}
%\begin{tabular}{lm{1.3cm}m{1.3cm}m{1.3cm}m{1.3cm}m{1.3cm}m{1.3cm}m{1.3cm}}
\begin{tabular}{lm{1.3cm}*{7}{>{\centering\arraybackslash}m{1.3cm}}}
\toprule
Modality & \multicolumn{2}{c}{\footnotesize{Cataract Surgery}} & \multicolumn{2}{c}{\footnotesize{OCT}} & \multicolumn{2}{c}{\footnotesize{MRI}} & \multicolumn{1}{l}{\multirow{2}{*}{\footnotesize{Avg. Rel.}}} \\ \cmidrule(lr){2-3}\cmidrule(lr){4-5}\cmidrule(lr){6-7}
Network & \footnotesize{DLV3+} & \footnotesize{scSENet} & \footnotesize{DLV3+} & \footnotesize{scSENet} & \footnotesize{DLV3+} & \footnotesize{scSENet} &   \\ \midrule
Supervised & 15.42 & 37.67 & 22.87 & 24.08 & 52.39 & 65.93 & N/A \\
$\Uppi$ Model~\cite{TESSL} & 27.55 & 35.56 & 1.12 & 0.00 & 10.00 & 6.87 & -22.88 \\
TE~\cite{TESSL} & 33.10 & 42.32 & 42.13 & 39.86 & 63.41 & 67.25 & 11.62 \\
Mean Teacher~\cite{UDAMIS} & 11.06 & 39.54 & 19.11 & 4.70 & 64.82 & 66.87 & -2.04 \\
RL~\cite{Reciprocal} & 34.40 & 45.13 & 48.73 & 47.70 & 60.79 & 70.20 & 14.77 \\
CPS~\cite{CPS} & 36.24 & 39.40 & 47.31 & 14.71 & \textbf{76.00} & 68.80 & 10.68 \\
ST~\cite{st++} & 34.34 & 41.10 & 36.84 & 33.01 & 68.63 & 71.97 & 11.26 \\\midrule
% ST++~\cite{st++} &  &  &  &  &  &  &  \\
{\bf TI-ST} & \textbf{37.69} & \textbf{45.31} & \textbf{50.93} & \textbf{40.87} & 66.56 & \textbf{74.07} & \textbf{16.18} \\ 
\bottomrule
\end{tabular}

\end{table*}


\cref{tab:quantitative} lists the performance of our transformation-invariant self-training (TI-ST) approach with alternative methods across the three tasks and using two network architectures. According to the quantitative results, TI-ST, RL, ST, and CPS are the best-performing methods. Nevertheless, our proposed TI-ST achieves the highest average relative improvement in dice score compared to naive supervised learning ($16.18\%$ average improvement). Considering our main competitor (RL), we note that our proposed TI-ST method is a one-stage framework using one network. In contrast, RL is a two-stage framework (requiring a pre-training stage) and uses a teacher-student network. Hence, TI-ST is also more efficient than RL in terms of time and computation.  Furthermore, the proposed strategy demonstrates the most consistent results when evaluated on different tasks, regardless of the utilized neural network architecture. 

\textfig{1}{figures/ablationf.pdf}{Ablation studies on the pseudo-labeling threshold and size of the labeled dataset.}{fig:ablation}

\cref{fig:ablation}-(a-b) demonstrates the effect of the pseudo-labeling threshold on TI-ST performance compared with regular ST. We observe that filtering out unreliable pseudo-labels based on transformation variance can remarkably boost pseudo-supervision performance regardless of the threshold. \cref{fig:ablation}-(c) compares the performance of the supervised baseline, ST, and TI-ST with respect to the number of source-domain labeled training images. While ST performance converges when the number of labeled images increases, our TI-ST pushes decision boundaries toward the target domain dataset by avoiding training with transformation variant pseudo-labels. We validates the stability of TI-ST vs. ST  with different labeling thresholds (0.80 and 0.85) over four training folds in \cref{fig:ablation_stability}, where TI-ST achieves a higher average improvement relative to supervised learning for different tasks and network architectures. This analysis also shows that the performance of ST is sensitive to the pseudo-labeling threshold and generally degrades by reducing the threshold due to resulting in wrong pseudo labels. However, TI-ST can effectively ignore false predictions in lower thresholds and take advantage of a higher amount of correct pseudo labels. This superior performance is depicted qualitatively in \cref{fig:qualitative}.

%\begin{figure}[b]
%\centering
%\includegraphics[width=1\textwidth]{figures/ablationf.pdf}
%\sidecaption{Ablation studies on the pseudo-labeling threshold and size of the labeled dataset. 
%}
%\label{fig:ablation}
%\end{figure}




%\begin{figure}[t]
%\centering
%\includegraphics[width=1\textwidth]{figures/ablation_stability10.pdf}
%\caption{Ablation study on the performance stability of TI-ST vs. ST across the different experimental segmentation tasks.
%}
%\label{fig:ablation_stability}
%\end{figure}

\textfig{1}{figures/ablation_stability10.pdf}{Ablation study on the performance stability of TI-ST vs. ST across the different experimental segmentation tasks.}{fig:ablation_stability}

%\begin{figure}[t]
%\centering
%\includegraphics[width=.9\textwidth]{figures/qualitative.pdf}
%\caption{Qualitative comparisons between the performance of TI-ST and four existing methods.
%}
%\label{fig:qualitative}
%\end{figure}
\widefig{1}{figures/qualitative.pdf}{Qualitative comparisons between the performance of TI-ST and four existing methods.}{fig:qualitative}
\section{Conclusion}
We have presented a method to locate markers in ETDRS rings for OCT scans by relying solely on slice-level annotations. By enforcing weak constraints on the loss function and modifying the pooling strategy of a standard convolutional network, we show that our method can learn to localize coarsely without annotations. To our knowledge, no other work has done so in the context of retinal imaging, and we have demonstrated that our approach achieves significant performance improvements over straightforward and state-of-the-art baselines. Further research will be focused on extending this to obtain per-pixel segmentation.