\graphicspath{{ch8_conclusion/}{Figures/}}

\chapter{Conclusion}
\label{chapter:conclusion}

% Contents: 
% Just conclude everything


\sidechaptersummary{Concluding Remarks}
%\desctotoc{Conclusion}

This dissertation sought to provide an optimistic view of the current field of Artificial Intelligence despite the challenges that it could bring to society and the economy. It focused on overcoming the scarcity of large, high-quality datasets in medicine by providing four solutions that addressed different aspects of the machine learning pipeline, from data acquisition and labeling to a real-world application of AI in a Reading Center.

Through the contributions summarized in the previous discussion, this dissertation has not only put into context the critical need for data\sidenote{See \Cref{chapter:introduction}}, but also provided comprehensive alternatives to mitigate the pressures associated with training data-hungry models. By situating these solutions within the broader context of technological transformation that AI encompasses, we underscore their significance and relevance in advancing the field. All facilitating a seamless transition into a new era.

As a colophon, I will bring back a paragraph from \Cref{chapter:introduction}. I said that ``negative implications [...] open the field to thoughtful solutions''\sidenote{See \nameref{subsec:potential_implications} in \Cref{sec:data_challenges}.}. The journey of technological advancement presents significant challenges, some of which I have covered in the present text. However, it also opens avenues for innovative solutions that can drive societal progress. The insights and solutions hereby presented offer a hopeful outlook, emphasizing that the benefits of AI can be harnessed to their full potential, ultimately contributing to a more advanced society. 

I do not want to conclude without offering a broader outlook. Even though this dissertation has centered around ophthalmology, the previous arguments can be applied equally to visual understanding in medicine in general. That said, I believe the future substitution of radiologists with AI models is greatly improbable. I envision a world where both systems will work in symbiosis, and AI will be a triage tool to rule out cases that do not need expert intervention. I think this world is still far away, and technology will have to improve not only performance-wise but also in terms of explainability. Hermetic, inscrutable machines as we have now will never be fully trusted either by the public or by fellow clinicians. At the same time, regulations will have to accompany technological progress to provide a legal framework in which both clinicians and patients are protected in unfavorable scenarios. It is good to have a hammer only if you know the difference between a nail and a screw.

