% Attention Mechanisms and Vision Transformers
%   Understanding Attention in Neural Networks
%   Vision Transformers (ViT): An Emerging Paradigm -> say also that they are not much better than convnets for medical things
%   Prompts for Vision Transformers - look at ECCV related work section

% Motivation: 
% feedforward networks and convnets: contribution of a value driven by its location.
% Attention: dynamically identify the parts of the signal that are more relevant --> translation.
%   Combine information from far away parts of the image. Discard information.
%   Features aggregated with an importance score that depends on the features themselves

\section{Attention Mechanisms and Vision Transformers}
\label{sec:attention}
Attention-based networks appeared in 2017\sideauthorcite{vaswani2017attention} and have come to stay. The Transformer, an architecture derived from the attention-based network, has become a widely used term\sidenote{GPT, the model under ChatGPT, stands for Generative Pre-trained Transformer.} in the context of a general public product utilized by 100 million users on a weekly basis\sideauthorcite{verge2023chatgpt}. At the time of writing, transformers are positioned as a near-term alternative to traditional neural networks, mainly due to their ability to scale efficiently, allowing training with millions of data points. All these advancements have perfused into computer vision, with more capable than ever models for image and video generation\sideauthorcite{blattmann2023stable} and scene understanding. Before diving into how this progress is shaping the future of artificial intelligence, we will discuss the motivation for a new kind of operation, different from the fully connected layers or convolutions we saw in the previous sections.

One of the main differences between a convolutional layer with respect to a fully connected layer is the receptive field\sidedef{Receptive field}{Region in the input space that affects a particular feature in the output space.}. In the former, each feature is affected by the features nearby, whereas in the latter, each unit receives information from the whole feature tensor. In both cases, however, the influence of an input value on the output depends entirely on their relative positions\sideauthorcite{attention_slides}. There exist tasks for which this arrangement is suboptimal: for instance, languages do not all share the same grammatical structures, and this has to be taken into account during translation\sidenote{To put an example, in German, the verb always comes in the second place, whereas in Japanese it is always at the end of the sentence.}. In the same context, words written at the beginning of a sentence can influence others far away. The lack of a processing method that dynamically identified the relevant structures for a given input motivated the development of attention-based models.

\subsection{Understanding the Attention Mechanism and its Implications}
The fundamental characteristic of the attention mechanism is the dynamic focus on different parts of the input. We will see now how this is achieved for 1-D sequences of length $T$ with $D$-dimensional elements in Scaled Dot-Product Attention as explained in \yeartextcite{vaswani2017attention} and shown in \Cref{fig:attention_diagram}. In the next section, we will generalize this to two-dimensional sequences (\ie~images).

Attention starts with three token\sidedef{token}{Elements into which a sequence can be divided.} sequences: queries ($Q\in \real^{M\times D_k}$), keys ($K\in \real^{N\times D_k}$) and values ($V\in \real^{N\times D_v}$)\sidenote{These names are not whimsical. Refer to \yeartextcite{graves2014neural} for historical context.}. The two first are combined to compute an attention matrix $A\in \real^{M\times N}$ which is weighted by $V$ to get the result sequence $Y\in \real^{M\times D_v}$:

\begin{equation}
    Y = \text{Attention}(Q, K, V) = \underbrace{\softmax\left(\dfrac{QK^T}{\sqrt{D}}\right)}_{A \in \real^{M\times N}}V
\end{equation}

\textfig[t]{1}{Figures/attention_diagram.pdf}{Scaled dot-product attention diagram as introduced in \cite{vaswani2017attention}. Diagram from \cite{abbott2024neural}.}{fig:attention_diagram}

Attention layers implement these operations with a previous step that linearly projects the input vectors into $Q$, $K$, and $V$\sidenote{$\linear$ refers to applying a fully connected layer without activation function.}:
\begin{eqnarray*}
    Q & = & \linear^q(X) \in \real^{M\times D}, \\
    K & = & \linear^k(Y) \in \real^{N\times D_k}, \\
    V & = & \linear^v(Z) \in \real^{N\times D_v}.
\end{eqnarray*}
Notably, sometimes $X\equiv Y\equiv Z$. In that case, this process is called \textit{self-attention}. In \Cref{chapter:samda}, we employ an attention layer in which $Y\equiv Z$. This is known as \textit{cross-attention}.

This kind of layer gave birth to the Transformer also in \yeartextcite{vaswani2017attention}. In that work, several attention layers were computed in parallel to form \textit{multi-head self-attention} and stacked in an encoder-decoder architecture\sidenote{In addition to attention, another noteworthy contribution of that work was the use of positional encoding, which ensured that the network received information about the relative or absolute position of the tokens in the sequence.}. This was tested on sequence-to-sequence translation, achieving state-of-the-art results at a fraction of the training cost. The basic architecture of the transformer encoder is illustrated on the right side of \Cref{fig:vit_architecture}.

\subsection{Vision Transformers (ViT): An Emerging Paradigm}
The initial formulation of transformers was designed for sequential data and, therefore, applied mainly to text. Researchers then began to explore the potential of transformers beyond text, considering their application to other types of data such as images. Nevertheless, this evolution proved more challenging than expected. Naively, one could flatten an image into a sequence of pixels and then apply attention layers to it. However, the storage requirements of this approach grow quadratically with the size of the images, as the dimensions of the attention matrix $A$ depend on the sequence lengths of the queries and the keys. Some initial works attempted to replace convolutions entirely with specialized attention patterns\sideauthorcite{ramachandran2019stand}, but these approaches did not scale effectively.

\textfig[t]{1}{Figures/vit_architecture.pdf}{Vision Transformer (ViT) architecture (left) treats an image as a sequence of smaller patches rather than individual pixels. These patches are linearly projected before being fed to the transformer encoder (right). Diagram from \cite{koleshnikov2021}.}{fig:vit_architecture}

Three years had to pass until the research community invented a successful interface of this kind of architecture to computer vision. \yeartextcite{koleshnikov2021} treats an image as a sequence of smaller patches rather than individual pixels. These patches are linearly projected before being fed to the transformer encoder, which is otherwise kept untouched with respect to the original one. This slight modification in data processing allows the transformer to handle the image more efficiently. It gave rise to a family of models now considered standard, as ResNet\sidecite{ResNet} is for convolutional architectures: the Vision Transformers (ViT). The left-hand side of \Cref{fig:vit_architecture} illustrates the basic model architecture of Vision Transformers. Rapidly, diverse modifications and evolutions of this architecture permeated into the traditional tasks of computer vision. Especially significant to the present thesis is the \autoindex{Segment Anything Model} (SAM)\sideauthorcite{sam}, which appended a decoder to the vision transformer to produce a promptable\sidedef{Prompting}{Providing specific input or queries to guide the model's output.} segmentation model. 

The advent of vision transformers did not necessarily result in enhanced performance out of the box\sideauthorcite{isensee2024nnu}. However, the highly efficient scalability of transformers paved the way for larger models. As previously discussed in \Cref{chapter:introduction}, scalability in deep learning is a double-edged sword: it can lead to remarkable generalization, but that comes at the expense of larger datasets and increased training costs. To illustrate, the largest models in \yeartextcite{koleshnikov2021}, with 632M parameters, demonstrated superior performance to their smaller counterparts (86M parameters) only when trained on JFT-300M, a proprietary dataset comprising 300M images\sidecitation{koleshnikov2021}{When pre-trained on the smallest dataset, ImageNet, ViT-Large models underperform compared to ViT-Base models [...] Only with JFT-300M, do we see the full benefit of larger models.}. Similarly, a comparable pattern was observed for convolutional models, which outperformed ViT on ImageNet but not on larger datasets.

This realization takes us back to the bitter lesson from \Cref{chapter:introduction}, whereby scalability, compute power, and data outperform a human-centric approach. At the same time, it brings additional difficulty in training models with such big datasets. Researchers have addressed these limitations in the medical domain by fully fine-tuning the models end-to-end, specifically for medical data. However, while this approach may offer a temporary workaround, the ongoing expansion of models and the scarcity of medical data present persistent challenges. These factors suggest that end-to-end training of future models may only be feasible for entities with ample resources.

\subsection{Prompting as a Novel Way of Training}\index{prompting}

In response to the challenges outlined in the section above, we have witnessed the emergence of \autoindex{Parameter Efficient Fine-Tuning} (PEFT) methods. These techniques aim to minimize the number of trainable parameters while achieving comparable performance to full fine-tuning. By training only a subset of parameters, PEFT methods aim to retain the knowledge of the base model when trained on a secondary task, thereby mitigating issues like \autoindex{catastrophic forgetting} and overfitting, especially when dealing with smaller target datasets. The landscape of PEFT methodologies is vast~\sideauthorcite{xu2023parameter}, with LoRA~\sideauthorcite{hu2022lora} standing out as a resilient method over time. LoRA incorporates two trainable low-rank matrices ($A$ and $B$) for weight updates and freezes the initial weights of attention layers $W_0$ during the update. The forward pass then results in the following:
\begin{equation*}
    h = W_0x + \Delta Wx = W_0x + BAx,
\end{equation*}
where $B\in\real^{d\times r}$ and $A\in\real^{r\times k}$ are the adaptation matrices with rank $r \ll min(d,k)$. 

This method is applied to the query, key, and value matrices within the attention layers in a transformer. Indeed, training the adaptation matrices reduces the expressiveness of the model but also increases the training efficiency due to the reduced number of parameters\sidenote{$W_0$ has $dk$ parameters, while $A$ and $B$ combined have $r(d+k)$.}.

Other works have envisioned adaptation methods that function as plugins for large models. These emerged along with large-scale models in the \autoindex{Natural Language Processing} literature~\sideauthorcite{houlsby2019parameter} with the Adapter framework and have subsequently spread to other fields of deep learning. The fundamental concept underlying these methods is to insert a module with few parameters into the base model and solely update those while maintaining the pre-trained model frozen. Initially pioneered by the Adapter framework, this approach inserts such modules sequentially after the self-attention layer in all transformer layers. Since then, various other methodologies have emerged, with adaptations in the position of the adapter.

Lastly, certain methods leverage the tokenization process of LLMs and implement PEFT with prompt tuning~\sideauthorcite{lester2021power}. One such method, LLaMA-Adapter~\sideauthorcite{llama_adapter}, was explicitly developed to fine-tune LLaMA~\sideauthorcite{llama}\sidedef{LLaMA}{Open source Large-Language Model from Meta.} into an instruction-following model. Specifically, in the higher transformer layers of LLaMA, they append a set of trainable, zero-initialized adaptation prompts as a prefix to the input instruction tokens. In \Cref{chapter:samda}, we will present an adapter for SAM based on LLaMA-Adapter, which shows superior performance in domain generalization.

% This evolution has also transpired into computer vision, precisely in semantic segmentation. Here, large models are beginning to establish their significance~\cite{chen2022vision,chen2023sam}. The medical domain presents formidable challenges and has witnessed recent advancements, exemplified by works like Medical SAM Adapter~\cite{wu2023medical}. More recently, \cite{ke2024segment} proposes a method to improve the quality of SAM segmentation masks via a learnable High-Quality Output Token injected into SAM's decoder that receives features from the ViT image encoder.

\sectionlinenew

Before exploring different learning strategies, we will redirect our attention from the latest trends in deep learning to a more traditional machine learning algorithm: Gaussian Processes. Although this might seem out of place, introducing this algorithm will be crucial for understanding \Cref{chapter:fullweak}, which will center around it.
