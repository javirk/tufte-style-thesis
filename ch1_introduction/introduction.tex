\graphicspath{{ch1_introduction/}{Figures/}}

\chapter{Data Challenges in AI}
\label{chapter:introduction}

% Sections: 
%Data Challenges in Computer Vision
%   Understanding the Challenge of Limited Data: talk about the evolution of data with respect to model size
%   Implications for Computer Vision Tasks: performance, generalization, data efficiency (methods that maximize the given data), robustness (look at wrong patterns)

%Overcoming Data Constraints
%   Importance of Data Augmentation and Synthesis: the first part is ok, the second maybe I remove
%   Transfer Learning and its Role in Leveraging Pretrained Models
%   Domain Adaptation: talk about what it is, types, history and examples

% Overcoming Budget Constraints
%   Weak annotations
%   Adaptive Annotation strategies
%   Active Learning: Not sure, it can be a very short part

%Thesis Statement: all the other subsections can be removed I guess
%%   Identifying Challenges in Data-Scarce Environments
%%   Proposed Solutions and Methodologies
%%   Contribution to Addressing Data Scarcity in Computer Vision

%Overview of Thesis Structure

\sidechaptersummary{Larger models need larger datasets, How to fight data and budget constraints, Thesis statement}
\desctotoc{Larger models need larger datasets --- How to fight data and budget constraints --- Thesis statement}

\subsubsection{Synopsis}AI and computer vision have benefited from rapid advances in technology and computing power, but these advances also present significant data challenges. As AI models become larger and more complex, their energy requirements and the need for extensive datasets increase. This poses two major hurdles: the monetary cost --- coupled with the environmental impact --- of training large models, and the difficulty of acquiring high-quality training data due to both size and budget. Data scarcity can lead to problems with model generalization, resulting in performance issues when AI systems are deployed in real-world scenarios. This has led to the development of several techniques that aim to mitigate the need for data by extracting as much information as possible from coarse labels.

%Data Challenges in Computer Vision
%   Understanding the Challenge of Limited Data: talk about the evolution of data with respect to model size
%   Implications for Computer Vision Tasks: performance, generalization, data efficiency (methods that maximize the given data), robustness (look at wrong patterns)

% The bitter lesson: Breakthrough progress eventually arrives by an opposing approach based on scaling computation by search and learning. The eventual success is tinged with bitterness, and often incompletely digested, because it is success over a favored, human-centric approach. http://www.incompleteideas.net/IncIdeas/BitterLesson.html
\section{Data Challenges in Artificial Intelligence and Computer Vision}\sidecitationnonum{sutton2019bitter}{Breakthrough progress eventually arrives by an opposing approach based on scaling computation by search and learning. The eventual success is tinged with bitterness, and often incompletely digested, because it is success over a favored, human-centric approach.}

The world is experiencing an unparalleled surge in technology usage and accessibility, driven by its constant evolution and the continuous decline in the prices of devices and services. Devices have become smaller and cheaper over time, which has made them accessible to a broader public. Furthermore, globalization and the Internet have intensified the interconnectedness of the world, thereby revolutionizing how humans access information and connect with each other. This has created new avenues for collaboration that have ultimately contributed and still contribute to the acceleration of progress.

\textfig[t]{1}{Figures/years_compute.pdf}{Computational capacity of supercomputers has doubled every 1.5 years from 1975 (not shown) to 2009, and this trend has been maintained in the last decade. Data from \yeartextcite{supercomputers2023dongarra}.}{fig:years_compute}

Simultaneously, the advancements in cost reduction and accessibility have been accompanied by a further significant enhancement in computing capabilities. Since 1971, when the first microprocessor was produced, the number of transistors in microprocessors has doubled every two years, following \autoindex{Moore's Law}\sideauthorcite{moore1998cramming}. Although this metric is not directly relevant to the end user, it has been replicated in other areas of interest to customers, such as speed and cost of computing. To put an example, the computational capacity\sidenote{Measured as the number of floating-point operations carried out per second, or FLOPS.} of supercomputers has doubled every 1.5 years from 1975 to 2009\sidecite{koomey2010implications}, as seen in \Cref{fig:years_compute}.

The field of Artificial Intelligence has not remained aloof from these advances but has greatly benefited from them. In 2005, the world witnessed the first hint of what the future could entail, when~\yeartextcite{steinkraus2005using} implemented a small artificial neural network\sidenote{I will present in \cref{sec:ml} a detailed description of neural networks.} on a \autoindex{Graphical Processing Unit} (GPU) and reported a three-times speedup over their CPU-based implementation. The work was followed by~\yeartextcite{chellapilla2006high}, which showcased the applicability of this technique in accelerating supervised convolutional networks. Until then, GPUs were specialized hardware components devised for video game rendering, a process that requires high parallelization\sidenote{To put it simply, each pixel operates independently of its neighboring pixels, so all of them can be rendered in parallel.}. The realization that artificial neural networks were formed by smaller components\sidenote{In \cref{sec:ml} we will learn that these small components are called ``neurons''.} that could be processed independently from others in the same layer connected the fields of graphics rendering and artificial intelligence, leading to said works. In the last decades, GPUs underwent a significant transformation, evolving from specialized gaming hardware to general-purpose processing units. Throughout this evolution, they have retained their most valuable feature, parallelization, while increasing computational capacity and memory constraints. This has resulted in a 1,000-fold increase in single-chip GPU inference performance over the past ten years\sideauthorcite{nvidia2023gpuperformance}.

\plainwidefig[t]{1}{Figures/model_compute.pdf}{Training computational requirements of machine learning models over time. Marker size represents the training dataset size; dashed lines are regressed only on the corresponding data in that period. Y-axis is logarithmic. Data from \yeartextcite{epoch2023pcdtrends}.}{fig:model_compute}

Artificial neural networks have profited from the above-described development and have grown in twenty years from two-layer fully connected networks in a \qty{256}{\mega\byte}-large GPU\sidecite{steinkraus2005using} to massively large models that require over \qty{3,000}{\giga\byte} of memory to be stored\sidenote{Number calculated with the most recent data for GPT-4. 1.8 trillion parameters at half-precision (FP16) require \qty{3520}{\giga\byte} of memory.}. However, such a frenetic growth in model size faces significant constraints stemming from energy consumption, monetary cost, and data availability. In the first place, the energy consumption of neural network training and inference processes poses a critical limitation, particularly as models scale up in size and complexity. Training large-scale neural networks demands substantial computational resources, leading to significant energy consumption and carbon emissions\sideauthorcite{luccioni2023estimating}. In the second place, the efficacy of these models is heavily reliant on the quality and quantity of the data on which they are trained. On the one hand, as the complexity of tasks tackled by neural networks expands, an ever-growing demand for larger datasets arises\sidecitation{hoffmann2022training}{We find that for compute-optimal training, the model size and the number of training tokens should be scaled equally.}.

On the other hand, ensuring the quality of said datasets requires extensive resources and expertise, which hampers the expansion of the datasets in question. This thesis focuses on this latter limitation and proposes solutions to address the constrained data availability.

\subsection{Limited Data into Numbers}
%\subsection{Understanding the Challenge of Limited Data}
To shed light on the discrepancy between the growth of models and datasets, we can analyze \Cref{fig:model_compute}, with data coming from \yeartextcite{sevilla2022compute}. From this, we can draw two primary conclusions: Firstly, the emergence of the first viable Deep Networks around 2010 significantly accelerated model growth, reducing the duplication rate from every 1.4 years to every six months\sidenote{From 1955 to 2010, model size increased steadily at the rate of 1.4x per year, following Moore's Law approximately. From 2010 to 2023, the rate was 4.1x per year.}. Secondly, dataset sizes previously tracked model size during the Pre Deep Learning Era, duplicating roughly every 20 months. However, they are currently lagging behind, with a duplication rate of every ten months. % This deceleration is particularly evident in datasets related to vision tasks.

\subsection{Potential Implications for Computer Vision Tasks}
% performance, generalization, data efficiency (methods that maximize the given data), robustness (look at wrong patterns) <-- this is dropped, maybe will add in the future
A model's learning capacity is directly proportional to the number of samples in the training set\sidenote{This growth is logarithmic for fully supervised training, as will be explained in \Cref{sec:fullweak_method}.}. Therefore, the most immediate consequence of limited data availability will be initially manifested as suboptimal performance on the corresponding testing set. However, we will see that the negative implications extend beyond low performance on in-distribution testing datasets.

Looking at the limited data problem from a different angle, a model may encounter difficulties extrapolating the training data when the training set fails to fully represent all aspects of a domain. This situation can result in unexpected failures when the model is deployed in production environments. In real-world settings, the model will be exposed to a multitude of scenarios that small training sets may not capture adequately, leading to a lack of generalization\sidedef{generalization}{Model's capability to adapt to unseen data, especially that coming from a different distribution.}.

Negative implications, as bad as they seem, open the field for thoughtful solutions. Generalization is a burden in small annotated datasets, but one can pose several meaningful questions: Would it be possible to increase this model's feature by using unlabeled data? Can one use the data as efficiently as possible to achieve the best result? The present thesis focuses on these types of solutions.
%Overcoming Data Constraints
%   Importance of Data Augmentation and Synthesis: the first part is ok, the second maybe I remove
%   Transfer Learning and its Role in Leveraging Pretrained Models
%   Domain Adaptation: talk about what it is, types, history and examples
\section{Overcoming Data Constraints}\label{sec:data_constraints}
Various strategies have emerged to address the data constraints and enhance the robustness and reliability of neural network systems. In this section, we briefly explore key approaches aimed at overcoming data limitations and maximizing the potential of neural networks.
\subsection{Importance of Data Augmentation} % and Synthetic Data
\autoindex{Data augmentation} stands as the epitome of data efficiency and is used worldwide to mitigate overfitting\sidedef{overfitting}{Behavior that a model can exhibit by which it captures the noise or random fluctuations in the data instead of its underlying patterns.} in machine learning models. This technique consists of modifying the data slightly so that the amount of samples virtually grows, and therefore the model is exposed to a wider variety. The introduction of slight modifications, furthermore, affects the quality of the data in the hope that it reproduces real-world scenarios more reliably. More often than not, datasets are curated, and for this reason, they contain pristine samples for each class. In traditional image datasets, objects will appear centered and undistorted, as seen in \Cref{fig:hrsod_examples} or in \Cref{fig:fullweak_datasets}. Opposingly, real-world images may display distortions that range from lens aberrations to noise, lower quality, or even differences in contextual information. All of which are important for understanding the content and meaning of the images.

\marginfig{Figures/hrsod_imgs.pdf}{Examples from HRSOD dataset~\parencite{hrsod_zeng2019towards} where the relevant objects appear centered and in the foreground.}{fig:hrsod_examples}

While data augmentation cannot solve all the above-stated issues, it can mitigate some of them. Focusing our study on the field of Computer Vision, data augmentation techniques may be separated into geometric, color space, and noise injection transformations.

Geometric transformations are mathematical bijective functions $\real^n \rightarrow \real^n$ that operate on points in n-dimensional space while preserving certain geometric properties, such as distances, angles, or parallelism. These transformations are typically described by functions or matrices and include operations such as translation\sidenote{
Translation of a point by $t_x$ and $t_y$ along the x and y axes:
\begin{equation*}
    \renewcommand*{\arraystretch}{0.8}
    \begin{pmatrix} 1 & 0 & t_x \\ 0 & 1 & t_y \\ 0 & 0 & 1 \end{pmatrix}.
\end{equation*}
}, rotation\sidenote{
Rotation of $\theta$ about the origin:
\begin{equation*}
    \renewcommand*{\arraystretch}{0.8}
    \begin{pmatrix} \cos\theta & -\sin\theta & 0 \\ \sin\theta & \cos\theta & 0 \\  0 & 0 & 1 \end{pmatrix}.
\end{equation*}
}, scaling\sidenote{
Scaling of $W$ and $H$ units:
\begin{equation*}
    \renewcommand*{\arraystretch}{0.8}
    \begin{pmatrix} W & 0 & 0 \\ 0 & H & 0 \\ 0 & 0 & 1 \end{pmatrix}.
\end{equation*}
}, or reflection\sidenote{Reflection may be treated as a special case of scaling, with $H=-1$ or $W=-1$ depending on the reflection axis.}.

Color space transformations operate on the pixel space and modify the color properties of images. Their formulation depends on the color space (RGB, HSV, Lab...), and they include transformations such as color jittering; brightness, saturation, or contrast adjustment; or converting the image to grayscale. Noise injection transformations, meanwhile, try to mimic real-world imperfections by injecting noise into images, while operating on the color space. Usually, this noise follows a Gaussian distribution and can be applied multiplicatively or additively.

Hereafter, we will refer to geometric transformations as \textit{``spatial transformations''}\index{spatial transformation} and to color space and noise injection as \textit{``non-spatial transformations''}\index{non-spatial transformation}. This is because the former transformations modify the spatial properties of the images, while the latter ones do not\sidenote{This issue will be particularly relevant in \Cref{chapter:tist}.}.

\subsection{Transfer Learning and its Role in Leveraging Pretrained Models}\index{Transfer Learning}
As it has been shown above, most modern models that achieve state-of-the-art (SOTA) performance do not fit into consumer GPUs \index{Graphical Processing Unit}, and therefore end-to-end training of these architectures from scratch is only feasible for organizations with ample resources. To get around this problem, researchers and end users worldwide rely on pretrained models from these organizations. These models have been trained on large datasets and thus already contain general features and patterns that are transferable to new, related tasks\sidedef{Task}{Combination of a task type (e.g. classification, semantic segmentation) and a dataset. \parencite{Mensink}}. The logic behind this is that a model trained on a large dataset will be able to adapt more quickly to a new setup, provided that the new task is semantically close. \yeartextcite{bozinovski1976influence} were the first to propose this idea and dubbed it Transfer Learning. Since then, the use of pretrained models in lieu of random initialization has become the de facto standard when training Deep-Learning models.

One common approach to transfer learning involves fine-tuning pretrained models. In this process, a pretrained model is first initialized with weights learned from the original task. Then, it is further trained on the new task with a smaller dataset, adjusting its parameters to fit the new task's requirements better. Fine-tuning allows the model to quickly adapt to the nuances of the new task while retaining the general knowledge acquired during pretraining. An alternative approach is feature extraction, whereby the pretrained model is employed as a feature extractor. Rather than modifying the parameters of the pretrained model, its internal representations are extracted and fed into a new, smaller model\sidenote{Be it a classifier, a segmentation head, etc.} tailored to the target task. Because the former approach implies training the whole model, the latter is particularly beneficial when the new task has a limited amount of data or when computational resources are constrained.

Transfer Learning has proved useful in a variety of scenarios, especially in settings where the amount of data is notably low, such as medical imaging. 

\subsection{Domain Adaptation, or How to Expand to Unseen Data}\index{domain adaptation}\label{subsec:da_intro}
%It is important to acknowledge that a task entails a task type and a dataset. While the modification of the task type 
Transfer learning relies on two fundamental assumptions. First, the pretraining and the new data must originate from similar distributions. Second, the pretraining corpus must be vastly larger than that of the target task. While the second assumption is usually fulfilled\sidenote{See previous section.}, the violation of the first is common for very task-specific datasets. This situation is described as \autoindex{domain shift}\sidedef{domain shift}{Scenario in which the distribution of two datasets is different, see \Cref{fig:domain_shift}.} and usually causes a decline in model performance due to a lack of generalization. Domain shift can occur for a variety of reasons, including changes in data over time, geographical shift\sidenote{Especially relevant for automated driving tasks, where visual cues vary depending on the region of the planet.}, or variations due to different contexts. In the context of vision tasks, the term ``domain shift'' encompasses a wide range of scenarios. These include images obtained with different devices or acquisition modes, as well as indoor versus outdoor images, and rainy versus non-rainy setups. Domain adaptation is a subcategory of transfer learning that aims to mitigate domain shift.

While mitigating domain shift, domain adaptation also has the potential to avoid the necessity of acquiring new annotations for related datasets. This challenge is particularly difficult to overcome in highly specific fields that require expert knowledge. In this line, several semi-supervised learning paradigms have emerged to reduce the need for extensive annotations in the \autoindex{target domain}. These approaches aim to leverage labeled and unlabeled data, allowing models to benefit from the unlabeled examples to enhance their performance without the high costs and labor associated with full annotation. 

\marginfig{Figures/domain_shift.pdf}{Domain shift example for two 2D distributions in 3D space. While both have the same mean and standard deviation, their covariance matrices are different. A model trained on the red distribution will most probably not adapt to the blue one.}{fig:domain_shift}

\Cref{chapter:background} will dive into the nuances of domain adaptation, exploring the different types in relation to the number of labeled samples in the target domain.

\sectionlinenew

As we have seen, addressing data limitations will be of paramount importance in order to scale machine learning models in the future. While initial measures have been implemented to alleviate the current constraints on this front, there is a further aspect of human progress that will also prove critical to the successful scaling of models: budget constraints.

% Overcoming Budget Constraints
%   Weak annotations
%   Adaptive Annotation strategies
%   Active Learning: Not sure, it can be a very short part
\section{Overcoming Budget Constraints}
Nothing is free, and this holds especially true in the field of machine learning. The formidable advancements that we have seen in computing power and model size (\Cref{fig:years_compute,fig:model_compute}) have led to a simultaneous increase in the required operational resources. As models become more complex, so do the resources they require - more powerful hardware, greater data storage, longer training time, and increased energy consumption. These escalating demands are having a direct influence on the costs incurred, both in financial terms and in terms of the environmental impact\sidenote{At the writing of this text, Meta has launched Llama 3, whose training has emitted \qty{8,930}{\tonne} of CO2 equivalent. For reference, a flight from Madrid to New York costs around \qty{1}{\tonne} per passenger.}. Moreover, the necessity for larger datasets, as previously discussed, is fueling the demand for human annotators who label all the data. These annotators are often employed in suboptimal conditions\sideauthorcite{time2023chatgpt} to counteract the rising costs.

While this may appear to be a problem in the long term, it is imperative to take action today in order to be prepared for what will inevitably occur. Even medium-sized companies will eventually be unable to train models from scratch, and they will have to rely on pretrained models. The problem will be exacerbated by the fact that a very small number of large corporations will have the capacity to control these pretrained models\sidenote{I believe only Open-Source alternatives will be able to counteract this trend that is already emerging.}. This will result in the unavoidable admission of the inherent biases these models will entail due to the limited number of people who will supervise their training. Consequently, research into methods of reducing costs at any stage of the machine learning pipeline\sidedef{Machine Learning Pipeline}{Workflow of a machine learning model encompassing all the required steps to put it into production. From data acquisition and labeling to deployment.} will be beneficial in the future.

Several methods have appeared to leverage data and annotations in this context, focusing on balancing costs and compelling practitioners to rethink how they approach machine learning development. This section explores this question, discussing the role of less-informative types of annotations and adaptive labeling strategies.

\subsection{Weak Annotations: Annotate Quickly, Analyze Carefully}\index{weak annotations}
Consider a scenario in which you are tasked with annotating many images for a computer vision project. Would you prefer to label each image with a single word, draw a box around each object, or use accurate polygons? Given the tedious nature of the annotating task, you would likely choose the one that requires the least time. Indeed, the process of labeling for image segmentation is more time-consuming than that of image classification, as one must classify each pixel semantically (as seen in \Cref{fig:weakannotations}). Formally, one can define the amount of information conveyed by an annotation $X$ using Shannon's formula:
\begin{equation}
    H(X)=-\sum_{x\in X} p(x)\log_b p(x).
\end{equation}

In the context of image classification, this implies that a label in a balanced dataset with 10 classes in a uni-label setting conveys approximately \qty{3}{\bit} of information. However, this figure is multiplied by the number of pixels for image segmentation, which is approximately \qty{33}{\kilo\bit} for a $100\times 100$ image\sidenote{For object detection, this number lies in the middle of both and depends on the number of objects. It is approximately \qty{282}{\bit} for five objects. This calculation is slightly more involved; for further details, refer to \Cref{app:bits}.}. 

\textfig[t]{1}{Figures/weakannotations.pdf}{Four different kinds of annotations. Namely, point, bounding boxes, polygons (also called coarse segmentation), and fine-grained segmentation. The images and segmentation annotations have been taken from the Cityscapes dataset~\cite{cordts2016cityscapes} and cropped for better visualization. Faces have been blurred.}{fig:weakannotations}

From a more practical perspective, \Cref{fig:supervision_prices} shows the market price and estimated labeling time for different types of annotations. It demonstrates that segmentation is significantly more costly than classification in terms of both time and money. Furthermore, it corroborates the previous idea that the greater the amount of information to be annotated, the greater the cost of the labeling work. For a similar reason, this cost is further exacerbated as soon as the domain requires expert annotators with great knowledge of the field. This is the case of medical annotations.

\textfig[t]{1}{Figures/supervision_prices.pdf}{Annotation time and price for different types of labels. The more information a type of annotation entails, the longer the labeling time and cost. Time data retrieved from \yeartextcite{Bearman16}, monetary costs from \yeartextcite{googlecompute}.}{fig:supervision_prices}

% Weak annotations are a form of data labeling where the provided information is less detailed or coarse in nature compared to traditional, fine-grained annotations. Unlike full\sidenote{Or strong.} annotations, which give explicit, pixel-level, or detailed region boundaries; weak annotations offer only basic cues about an image's content.  

Returning to the initial question, it appears that image-level annotations, such as classification, would be the preferred option due to the reduced amount of work involved. However, this would result in segmentation tasks, which are highly labeling-greedy, being disregarded in traditional settings that require full-level supervision. In contrast, weak annotations\sidedef{weak annotations}{Data labels that provide coarse or general information about a data point. They are less complex than the final task and are therefore designated as ``weak''.} differ from traditional supervision in the amount of information provided, offering a broader or less granular level of labeling. They provide sufficient information to facilitate segmentation without the expense and labor involved in traditional strong annotations, although the training method must be modified. In this respect, image-level labels, bounding boxes, points, or scribbles are considered weak annotations for the segmentation task, as the latter requires more information.

The increasing availability of weak annotations is providing a powerful alternative to full annotations, particularly in the field of computer vision. In recent times, text annotations and image descriptions without specific labels have emerged as a valuable resource for model pretraining in frameworks such as CLIP\sideauthorcite{radford2021learning}. 

\subsection{Adaptive Annotation Strategies and Active Learning}\index{adaptive annotation}\index{active learning}
Effective annotation is key to building robust models. As previously discussed, weak annotations entail less precise or partially labeled data, which can be beneficial for model training but only when used effectively. The introduction of these annotations has opened the door to alternative training methods. However, the challenge lies in maximizing their utility without compromising model accuracy. 

Adaptive annotation strategies and active learning have emerged as effective approaches to address this challenge. Adaptive annotation involves adjusting the annotation process dynamically to focus on the most beneficial annotation type or data point. This flexibility allows for the efficient use of weak annotations by concentrating resources where they can have the most significant impact. This is typically achieved via a policy that acts iteratively and evaluates the most beneficial annotation strategy for the model at each iteration. A more detailed examination of this topic is presented in \Cref{chapter:fullweak} of this work, in which one such policy is also proposed.

For its part, active learning\sidedef{Active Learning}{Learning paradigm in which the algorithm iteratively queries the user to label new data points based on their potential learning benefits.} is a type of adaptive annotation strategy that focuses on data sampling. In this case, the annotation type remains unchanged, and the policy's objective is to select the data points that will facilitate the greatest improvement in the model's performance in the next iteration. Consequently, human annotator efforts are directed toward the most informative samples, thereby reducing the overall annotation workload while maintaining model robustness.


%Thesis Statement
\section{Thesis Statement}
The goal of this dissertation is to address the previous challenges from different perspectives, considering the machine learning pipeline as a set of parts that can be optimized in terms of data efficiency. As discussed above, the data challenges are particularly severe in vision tasks applied to the medical domain, where the amount of data is even more limited due to the necessity of expert annotators. We will, therefore, focus on the medical domain. This leads to the following thesis statement:
\\%[12pt]

\textit{The surging demand for larger datasets, which is promoted by ever-larger models and the advancements in computing, can be counteracted by making informed labeling choices, carefully allocating the budget, and leveraging field-specific constraints to the model's advantage.}
\\

We elaborate on each claim in turn.
%Overview of Thesis Structure
\section{Organization}
\Cref{chapter:background} provides the technical background that informs the rest of the dissertation. It begins with an introduction to the fundamentals of machine learning and deep neural networks \cref{section:neural_networks} for vision tasks with a special emphasis on the mechanism of attention layers \cref{section:attention}, their integration into Vision Transformers \cref{section:vision_transformer}, and the implications of this on model training evading end-to-end fine-tuning. It then shifts to other machine learning topics, including Gaussian Processes \cref{section:gaussian_process} and the presentation of non-traditional training strategies such as semi-supervised training \cref{section:semi_supervised} and several domain adaptation flavors \cref{section:domain_adaptation}. The final section explores concepts in ophthalmologic image acquisition that will be needed throughout the dissertation.

\Cref{chapter:oct} focuses on the medical aspects of data usage and addresses one of the final stages of the machine learning pipeline, namely the application of a model in a practical setting. The chapter proposes a method for locating biological markers from Optical Coherence Tomographies that only requires weak annotations. It also portrays how to leverage domain knowledge to formulate a loss function that constrains the output \cref{sec:oct_method}. Finally, the presented method is tested on a variety of settings \cref{sec:oct_results}.

\Cref{chapter:fullweak} manages an earlier task in the pipeline, namely the data acquisition process from the cost allocation perspective. This chapter explores the possibility of mixing different kinds of annotations by defining a strategy that aims to maximize performance with budget constraints. It puts the concept of Gaussian Processes into use to extrapolate model performance with unseen data \cref{sec:fullweak_method} in an iterative manner. Subsequently, it is demonstrated that an adaptive strategy can achieve near-optimal performance for a range of annotation budgets and datasets.

\Cref{chapter:tist} tackles the domain adaptation problem for image segmentation from the unsupervised point of view. Specifically, it addresses the problem of overcoming distribution shifts between different imaging devices and configurations in the clinical s+--etting. It does so by employing a self-training technique that leverages the fact that neural network performance should be invariant to image transformations \cref{tist_methodology}. The proposed method, designated as TI-ST\sidenote{Acronym for Transformation-Invariant Self-Training}, is subjected to evaluation in three medical image modalities in comparison with several state-of-the-art domain adaptation methods \cref{sec:experimental_results_tist}.

\Cref{chapter:samda} further develops the domain adaptation problem, albeit from a different perspective. As previously discussed, end-to-end fine-tuning of models is turning computationally intractable. This chapter introduces a novel adapter for one of the most well-known semantic segmentation architectures that minimizes the number of trainable parameters while maintaining the performance of the base model. 

\Cref{chapter:discussion} presents a summary of the contributions made by this thesis, a discussion of the findings and shortcomings, and an outlook on future work.