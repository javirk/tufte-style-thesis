\section{Introduction} \label{sec:intro}
%\subsection{Linear surface features on Europa}
% introduce bands, rc, db, ul here?

Different forms of lineaments may evolve out of initial cracks in the icy surface of Jupiter’s moon Europa\sidecite{Figueredo2004,Bradak2023b,Greenberg1998,ProckterPatterson2009}. Therefore, all types of lineaments have the potential to link to subsurface liquid water, as evidence is found for Enceladus\sidecite{Hansen2006,Sladkova2021,Ingersoll2016}. This makes lineaments of high interest to Europa's potential as a habitable world\sidecite{Daubar2024}.



The Solid State Imager (SSI) onboard NASA’s Galileo spacecraft mapped approximately 10\% of Europa’s surface at a regional resolution of 200-230 m/px. Two north-south covering image mosaic swaths acquired under similar illumination conditions portray parts of the leading and trailing hemispheres (in the following abbreviated as ``RegMaps'', see \Cref{fig:RegMaps_basemap}). Because of the north-south extent and leading/trailing locations, these image mosaics enable the study of regional geological processes across a range of latitudes and longitudes\sidecite{Figueredo2003, Sarid2004,Patterson2006,Collins2022}. However, they do not enable features to be linked globally. 
Unfortunately, the complete existing geologic maps\sidecite{Figueredo2004} of the RegMaps are not available in a digitized version. Furthermore, the level of detail with respect to the mapped lineaments is insufficient for an extensive lineament analysis due to the scale of the map. The map we present in this work is at a larger scale and therefore reveals fine lineae in areas previously mapped as ``ridged plains''\sidecite{Greeley2000}.

\plainwidefig[t]{1}{regional_mosaics/Figures/basemap_excerpt.png}{Global map with regional maps (RegMaps) in shades of red and the region for the manually revised lineament map (red box). Top: Basemap from USGS~\cite{USGS2002}. Bottom: Global geologic map~\cite{Leonard2024}.}{fig:RegMaps_basemap}

The surface of Europa is characterized by tectonic features overlaid by a small number of impact craters and disrupted in some regions by chaotic terrain. Regional and global mapping has identified three broad episodes of deformation\sidecite{Prockter1999,Figueredo2000,DoggettEuropaBook2009,Leonard2024}. The oldest identifiable tectonic unit, ridged plains (\Cref{fig:features_exs}.A), is widespread across the surface. Characterized by multidirectional, mostly high-albedo linear ridges, the ridged plains unit is overlaid by all later structures such that it appears as background plains material\sidecite{Daubar2024,Leonard2024}. Next in the stratigraphic sequence are bands, linear or curvilinear dilational features that have formed by symmetrical spreading from a crack or double ridge with new subsurface material filling the gap\sidecite{Greeley2000,Tufts2000}, analogous to mid-ocean ridges on Earth\sidecite{Sullivan1998} (\Cref{fig:features_exs}.B). Band morphologies fall into two major types: relatively smooth material comprised of small hummocks, and more rugged subparallel ridges\sidecite{Prockter2002}, that may be related to the band’s opening rate\sidecite{Stempel2005}. The youngest episode of deformation is the result of ``chaos'' formation, in which the surface has been disrupted from below, resulting in the breakup of preexisting terrain into discontinuous blocks of icy material set into a darker, hummocky matrix (\Cref{fig:features_exs}.C)\sidecite{Greeley2000}. Microchaos consists of smaller, discontinuous subcircular patches of chaos\sidecite{Pappalardo1998,Spaun2002,Noviello2019}, whose relationship to the larger chaos regions is not yet understood\sidecite{Collins2009}. Chaos may result from upwelling diapirs of thermally or compositionally buoyant material intersecting the surface\sidecite{Greenberg1999,Sotin2002,Figueredo2002,Schmidt2011,Collins2009} and it has been proposed that a periodic thinning and thickening of the ice shell drives periodic formation of chaos and regional plains terrain on a timescale comparable with the inferred surface age of 20 to 200 Myrs\sidecite{Bierhaus2009,Hussmann2004}. The regional plains unit makes up for 53\% of the global coverage, which makes it the most common unit, followed by chaos terrain with 40\%\sidecite{Leonard2024}.

\plainwidefig[t]{1}{regional_mosaics/Figures/Features_examples_grid.png}{A selection of Europan surface type features.}{fig:features_exs}

One feature that has formed throughout Europa’s stratigraphic history is the double ridge, which is widespread across the surface\sidecite{Figueredo2000}. Consisting of a distinct V-shaped trough flanked on each side by a single ridge\sidecite{Head1999,Greeley2000} (\Cref{fig:features_exs}.D), double ridges are observed at a range of size scales and as linear or curvilinear features a few tens of meters to hemispherical in length, and some forms appear related to diurnal tides\sidecite{Hoppa1999a}. Multiple models have been proposed for their formation, however, none are able to fully match all the observations\sidecite{ProckterPatterson2009,Daubar2024}. Ridge complexes are less common and consist of several adjacent parallel or anastomosing sets of double ridges\sidecite{Greeley2000} (\Cref{fig:features_exs}.E). Their formation mechanisms remain obscure\sidecite{Kattenhorn2009}. The youngest tectonic features on Europa are troughs, which appear similar to the V-shaped troughs of double ridges but lack the flanking ridges\sidecite{Greeley2000} (\Cref{fig:features_exs}.F). These are not observed in older terrains, perhaps because they fade and are overlooked easily\sidecite{Greeley2000}, but it is likely that they have been exploited to form other types of tectonic features\sidecite{Kattenhorn2009}. 
 
Because so little is known about how Europa’s tectonic features relate to the stresses that form them, including diurnal, non-synchronous, and polar wander\sidecite{Kattenhorn2009,Sotin2009}, the extent to which the tectonic landforms penetrate the ice shell is not well understood. Knowledge of the pathways that these features create or exploit is critical to understanding how material moves through the ice shell, and whether liquid water, including the ocean, is accessed. 
% End of Louise's paragraph

A global geologic map\sidenote{Such as the one in~\cite{Leonard2024}, or \Cref{fig:RegMaps_basemap}}, provides the global context for current missions to Europa, namely JUICE and Europa Clipper. The Europa Imaging System (EIS) on NASA's Europa Clipper will deliver a global view of Europa at a resolution $\leq$100 m/px in the 2030's\sidecite{TurtleSSR}. This global dataset will be valuable for lineament cross-cutting analyses and can, for the first time, result in a relative dating of globally interlinked regions\sidecite{Daubar2024}. The relative ages can be compared with absolute age results from crater counting\sidecite{Daubar2024}. Already for this reason alone, a global lineament map is valuable. Furthermore, spatially varying lineament characteristics, for example a regionally varying width of lineaments or different categorical distributions can be used to interpret the geological evolution of Europa's surface.

However, at 100~m/px, lineaments in background ridged plains appear as a dense network, which makes a complete lineament mapping time-consuming. This is where deep learning algorithms can help. \sidecite{Haslebacher2024a} have shown that an automated mapping framework with human validation reduces the workload while still accurately mapping 30-50~\% of lineaments. They trained a regional convolutional neural network, or Mask R-CNN\sidecite{He2018} for lineament mapping in Galileo images showing Europa's surface. Although the time used for mapping can be reduced with LineaMapper v1.0, the tool is not optimized for the resolution and illumination conditions of the RegMaps and would benefit from a larger training dataset. 
To maximize scientific return and flexibility in planning with EIS, an early preparation based on existing imagery is needed to ensure that the evaluation and analysis of all images is optimal. For this, we aim to improve LineaMapper's ability to correctly detect and segment a higher number of lineaments and to provide more continuous segmentation masks.

\plainwidefig[t]{1}{regional_mosaics/Figures/2024_07_17_apply_LineaMapper_scheme_optimised-01.png}{Workflow for assembling LineaMapper's predictions. A moving window tiling algorithm ensures overlapping areas for stitching the predictions together, which are based on smaller tiles of a context image. The need for such an algorithm is especially high for lineaments, which oftentimes run over an area larger than the tile size.}{fig:graphic_Stitching_algo}

Here, we attempt to map all identifiable lineaments in the Galileo RegMaps for an analysis of the evolution of lineament characteristics throughout their formation. This includes lineaments in the ridged plains, which have not frequently been mapped previously, but if mapped, a stratigraphy of lineaments inside ridged plains is insightful\sidecite{Bradak2023}.
To achieve this, we use a new methodology: We use the automated mapping tool LineaMapper\sidecite{Haslebacher2024a} together with a dedicated stitching algorithm to guide the mapping of the RegMaps iteratively. With 2140 new lineaments from a revised part of the RegMaps\sidenote{See red box in \Cref{fig:RegMaps_basemap}}, we train and release LineaMapper v1.1 and v2.0, which are optimised for the RegMaps. Subsequently, the complete Galileo RegMaps are fed into LineaMapper v1.1 and v2.0 and automatically assembled into a continuous lineament map of the RegMaps.

We extract the length, the width, the azimuth, and the fragmentations per kilometer (FPK), which we use as an estimation of lineaments relative age. With these extracted characteristics, a number of questions have the potential to be answered, which we explore in this work: 

\begin{enumerate}
    \item Is there a difference in the number or area density of lineaments in the chaos vs. ridged plains region?
    \item Does the data suggest that the distribution of lineament width, length, number of fragmentations, and FPK, for each lineament category is equal within chaos and ridged plains?
    \item Do we find correlations between lineament characteristics, and do the fits vary for the chaos and ridged plains unit?
    \item Can LineaMapper identify different terrain types (chaos and ridged plains) by lineament density?
\end{enumerate}

\textfig[t]{1}{regional_mosaics/Figures/galilean_moons_2023_Marseille_LineaMapper_workflow_circle.pdf}{Iterative cycle for improving LineaMapper and generating a near-complete lineament map.}{fig:circle_workflow}
